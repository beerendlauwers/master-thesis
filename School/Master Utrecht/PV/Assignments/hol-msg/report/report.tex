\documentclass[10pt,a4paper]{article}
\usepackage[latin1]{inputenc}
\usepackage{amsmath}
\usepackage{amsfonts}
\usepackage{amssymb}
\usepackage{listings}
\author{Beerend Lauwers and Frank Wijmans}
\title{Report Project 1: Social Networks privacy verification}
\begin{document}
	
	\maketitle
	
	\section{Introduction}
	
	
	\section{Types}
	
	Users are identified by numbers; $ User : num $.
	Users follow other uses as described in a network.
	Users can send messages to others according to some privacy constraint.
	
	Equally as for users, constraints are identified by numbers in the following order:
	\begin{enumerate}
		\item 0 Me - The message is for the sender only.
		\item 1 Fo - The message is for the followers of the sender and the sender.
		\item 2 Fo2 - The message is for followers of followers, followers and the sender.
		\item 3 All - The message is for everybody.
	\end{enumerate}
		
	For the network, we use a set of tuples of $ num $. 
	The definition of the type is $ Network : ((num\#num) set) $.
	The num values in the tuple correspond to $ User $ identifiers. 
	The first is the follower and the second the followee. 
	The network consists of one way relations.


	A message consists of a triple, with the first two values being a sender and receiver in the form of $ User $  identifiers, together with a privacy constraint as stated above (0-3).
	Our message data consists of a set of these triples: $ Data : ((num\#num\#num) set) $.
	To check safety or correctness, the assumption is that all sent messages are in $ Data $, and nothing more.

	
	\section{Algorithm}
	
	A correctly functioning algorithm must do two things: check the privacy constraint of a message, and construct messages that comply to these privacy constraints and add them to the existing message data.

	The privacy constraints are as follows:
	\begin{itemize}
		\item Given $ constraint == 0 $, the receiver and sender should be equal.
		\item In the case of $ constraint == 1 $, the receiver and sender should be equal or the receiver follows the sender.
		\item If $ constraint == 2 $, then the same for constraint 1 holds, or the receiver is a follower of some follower of the sender.
		\item And in the last case, everyone is allowed to receive the message.
	\end{itemize}

	An incorrectly functioning algorithm will not respect these constraints, and should not pass safety checking.
	
	\section{Definition of safe}
	
	Safety is defined by inspecting the message data, namely that the receiver and sender have a certain relationship as described by the privacy constraint, which is also available in the message:

	\begin{lstlisting}[mathescape=true]
val safe_def = 
Define `safe (network:(num#num) set) (dataInp:(num#num#num) set) = 
$\forall$ y orig cons. (y, sender, cons) IN dataInp $ \Longrightarrow $ 
((cons=0) $ \Longrightarrow $ (y=sender)) $ \wedge $ 
((cons=1) $ \Longrightarrow $ ((y=sender) $ \vee $ ((y,sender) IN network))) $ \wedge $ 
((cons=2) $ \Longrightarrow $ ((y=sender) $ \vee $ ((y,sender) IN network) 
  $ \vee $ ($\exists$ y x. (y,x) IN network $ \wedge $ (x,sender) IN network))) $ \wedge $ 
((cons=3) $ \Longrightarrow $ (T)) `; 
	\end{lstlisting}

	Checking if an algorithm respects the privacy constraints is then straightforward:
	If the existing message data is safe, the new message data created by the algorithm, combined with the existing message data, must also be safe:

\begin{lstlisting}[mathescape=true]
checkSafety algo sender receiver constraint =
$ \forall $ network data. 
   safe network data $ \Longrightarrow $ safe network (algo data sender receiver constraint)
\end{lstlisting}

	We can then quantify over the constraint, sender and receiver to verify that our algorithm is safe in all cases:

	\begin{lstlisting}[mathescape=true]
$ \forall $ sender receiver constraint. 
   checkSafety algorithmDef sender receiver constraint
	\end{lstlisting}
	
\end{document}