\documentclass[10pt,a4paper]{beamer}
\usepackage[utf8]{inputenc}
\usepackage{amsmath}
\usepackage{amsfonts}
\usepackage{amssymb}
\author{Bruce Edmonds (2002)
\newline
\newline
Paper presented by Beerend Lauwers}
\title{How Formal Logic Can Fail to be Useful for Modelling or Designing MAS}

\begin{document}
	
\frame{\titlepage}

\begin{frame}{Outline}
\tableofcontents
\end{frame}

\section{Introduction}
\subsection{Motivation}

\begin{frame}{Bruce Edmond's motivation}
\begin{itemize}
\item Formal systems are only a tool, not the content
\item Choosing the wrong system introduces biases
\item New formal systems should prove themselves!
\end{itemize}
\end{frame}

\subsection{Some History}

\begin{frame}{Some History}
\begin{itemize}
\item 1962: Whitehead and Russell prove that set theory can be formalised with first-order predicate logic
\item Many different logics emerge, with two camps:
\begin{itemize}
\item Philosophical approach
\begin{itemize}
\item Concentration on axioms of the logic
\item Proof theory and formal semantic "more of an afterthought"
\item Intuitionistic logic, Free logic, Relevance logic, Modal logic
\end{itemize}
\item Pragmatic approach
\begin{itemize}
\item Extensions of first-order predicate logic
\item Practically-oriented: Semantics checkable, easy to model, inference, do computation with?
\item SDML as an example
\end{itemize}
\end{itemize}
\item First approach appears more popular and AI and MAS
\item Logics are compared and discussed based only on a small set of properties
\item Edmonds argues against this approach: not useful for understanding or building MAS!
\end{itemize}
\end{frame}

\section{Problems with generalization}

\subsection{Generalization of logics}

\begin{frame}{Generalization of logics}
\begin{itemize}
\item General theories appear to be preferred
\item Three common ways to generalize (and their costs!):
\end{itemize}
\begin{enumerate}
\item Abstract away from details, only look at broad domain truths \textbf{(post-hoc abstraction)}
\begin{itemize}
\item Loss of information by ignoring particular cases!
\item Information may be crucial to your research goal!
\end{itemize}
\item Determine structure beforehand, ignore contradicting cases \textbf{(a priori abstraction)}
\begin{itemize}
\item Loss of relevance!
\item May exclude what you're researching!
\end{itemize}
\item Allow for adaptation to particular cases \textbf{(adaptive generality)}
\begin{itemize}
\item Computationally expensive: may be unrealistic!
\end{itemize}
\end{enumerate}
\begin{itemize}
\item Philosophical approach tends to generalize fruitlessly
\end{itemize}
\end{frame}

\begin{frame}{Edmond's arguments against needless generalization}
\begin{itemize}
\item Increased generality isn't a necessity
\item Thinking up new logics is usually a better idea
\item OTOH, choosing an incorrect specific formal system biases development of one's theory!
\item \textbf{Conclusion:} Comparing systems by their level of expressivity is a weak justification
\begin{itemize}
\item The generalized version also supports it
\item Easy to go wrong with it
\end{itemize}
\item But of course, we still need these formal systems!
\end{itemize}
\end{frame}

\subsection{High-level theories}

\begin{frame}{High-level theories}
\begin{itemize}
\item Highly generalized  logics aren't the way to go
\begin{itemize}
\item Too general to be useful
\item Too complex to be insightful
\end{itemize}
\item Any such proposed 'high-level' theories should be scrutinized
\item Intermediate levels of abstraction work very well, too
\item Established theories and methods are still up in the air
\item Some papers seem to think otherwise, assuming their foundations to be established and proven
\end{itemize}
\end{frame}

\section{Problems with the type of logics}

\subsection{What logics are suitable?}

\begin{frame}{What logics \emph{are} suited for MAS?}
\begin{itemize}
\item In the past:
\begin{itemize}
\item Non-temporal
\begin{itemize}
\item Example: BDI logic's temporal element is implicit
\item Using a single state (timewise) is bound to lead to problems
\end{itemize}
\item Context-independent
\begin{itemize}
\item Reasoning about norms, goals, etc. requires knowledge of contexts
\end{itemize}
\item Propositional
\begin{itemize}
\item Most proposed logics do not support numbers
\item G\"{o}del’s incompleteness theorem is the culprit here
\item But we can never prove all of a MAS' properties anyway!
\end{itemize}
\item Lack of formal semantics
\begin{itemize}
\item Philosophical approach emphasises 'meaning'
\item Surely formal semantics are important, then?
\end{itemize}
\end{itemize}
\item Abstracting away from time, context and numbers to get a general theory?
\begin{itemize}
\item Onus is upon the authors that it is valuable
\end{itemize}
\item Better would be to use:
\begin{itemize}
\item Temporal
\item Contextual
\item Predicate
\end{itemize}
\item More complex, but MAS aren't simple systems!
\end{itemize}
\end{frame}

\subsection{The audience}

\begin{frame}{The audience}
\begin{itemize}
\item People first want to know if a work is worth learning: communication is paramount!
\item Formal logic is claimed to aid communication
\item It may be precise, but aiding in communication?
\begin{itemize}
\item Logics require some knowledge to understand
\end{itemize}
\item Most people are used to this kind of approach, aiding acceptance of papers
\item Some of the papers were simply unproved ideas and intuitions
\item Papers that do not show the usefulness of a formalism prevent the audience from evaluating it properly
\item Using a formal system to model a realistic MAS will prove more valuable than intuition-based papers
\end{itemize}
\end{frame}

\subsection{Arguments for formalism}

\begin{frame}{Arguments for formalism}
\begin{itemize}
\item Edmonds doesn't want to inhibit experimentation with new logics
\item But the relevance of new formal systems has to be demonstrated
\item Some papers don't solve particular problems but also refrain from formal theorems and proofs - unacceptable!
\item Cheap computational power allows for quicker experimentation
\end{itemize}
\end{frame}

\section{How to go forward?}

\begin{frame}{How to go forward?}
More pragmatic approach:
\begin{itemize}
\item Begin with less abstract models, chain models of different levels of abstraction together
\item Sceptical view at unproved abstract theories
\item Similarly for papers based on intuition and nothing else
\item Papers suggesting formal systems for MAS should provide a demo MAS
\end{itemize}
A 'shortcut' to a powerful high-level theory (to bypass the empirical work required) may be nice, but is unrealistic and doesn't provide the necessary chain of abstractions to guide further search.
\end{frame}
	
\end{document}
		