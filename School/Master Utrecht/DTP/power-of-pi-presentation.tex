\documentclass[10pt]{beamer}
\usepackage[latin1]{inputenc}
\usepackage{amsmath}
\usepackage{ stmaryrd }
\usepackage{amsfonts}
\usepackage{amssymb}
\usepackage{listings}
\usetheme{Antibes}
\lstset{language=Haskell}
\lstset{breaklines=true}

\newcommand\doubleplus{+\kern-1.3ex+\kern0.8ex}
\newcommand\mdoubleplus{\ensuremath{\mathbin{+\mkern-10mu+}}}

\newcommand\monadbind{\gg\!\!=}

\author{
	Nicolas Oury and Wouter Swierstra\\
	\ \\
	\ \\
	\
	Presented by Beerend Lauwers\\
	\
	Utrecht University, The Netherlands
}
\date{\today}
\title{The Power Of Pi}

\begin{document}

\frame{\maketitle}

\begin{frame}{Outline}
\tableofcontents
\end{frame}

\section{Aim of the paper}
\begin{frame}{Aim of the paper}

\begin{itemize}
\item Dependently-typed programming matters!
\item Three case studies to exemplify this:
	\begin{itemize}
	\item Cryptol (DSL)
	\item Data Description Languages
	\item Relational Algebra
	\end{itemize}
\item In each case study, commonly-used DTP concepts are introduced as they are required.
\item Agda is used as the dependently-typed language.
\end{itemize}

\end{frame}

\section{Case studies}
\subsection{Cryptol}
\begin{frame}[fragile]{Cryptol - Overview}

\begin{itemize}
\item DSL for cryptographic protocols, developed by Galois with help from the NSA.
\item Built for high-level descriptions of low-level cryptographic algorithms.
\item Two distinguishing features:
	\begin{itemize}
	\item Word length is recorded in the type:
	\begin{lstlisting}
	x : [8];
	x = 42;
	\end{lstlisting}
	\item Special pattern-matching for splitting words into pieces:
	\begin{lstlisting}[mathescape=true]
	swab : [32] $ \rightarrow $ [32];
	swab [a b c d] = [b a c d]; -- 4 words of 8 bits
	-- [a b] would have pattern-matched on 2 words of 16 bits
	\end{lstlisting}
	\end{itemize}
\item Not embedded! Has its own interpreter and compiler.
\item Let's try and replicate this in Agda.
\end{itemize}

\end{frame}

\begin{frame}[fragile]{Cryptol - Cryptol's types in Agda}
\begin{itemize}
\item 
	\begin{lstlisting}[mathescape=true]
	data Bit : Set where
	 O : Bit
 I : Bit
	Word : Nat $ \rightarrow $ Set
	Word n = Vec Bit n
	\end{lstlisting}
\item Great, but how can we define that special pattern matching behaviour?
\item First DTP concept: \textbf{Views}.
\end{itemize}
\end{frame}

\begin{frame}[fragile]{DTP concept: Views}

\begin{itemize}
\item Views = defining custom pattern matches
\item Example: SnocView : recursing over a list, reversed.
\item First, define a datatype:
	\begin{lstlisting}[mathescape=true]
	data SnocView {A : Set} : List A $ \rightarrow $ Set where
	Nil : SnocView Nil
	Snoc : (xs : List A) $ \rightarrow $ (x : A) $ \rightarrow $ SnocView (append xs (Cons x Nil))
	\end{lstlisting}
\item Second, define a conversion function:
	\begin{lstlisting}[mathescape=true]
	view : {A : Set} $ \rightarrow $ (xs : List A) $ \rightarrow $ SnocView xs
	view Nil = Nil
	view (Cons x xs) with view xs
	view (Cons x $ \lfloor $Nil$ \rfloor $) | Nil = Snoc Nil x
	view (Cons x $ \lfloor $append ys (Cons y Nil)$ \rfloor $) | Snoc ys y = Snoc (Cons x ys) y
	\end{lstlisting}
\end{itemize}

\end{frame}

\begin{frame}[fragile]{DTP concept: Views}

\begin{itemize}
\item To apply the custom pattern match, we apply it to a list.
\item We can then use the result as usual:
	\begin{lstlisting}[mathescape=true]
	rotateRight : {A : Set} $ \rightarrow $ List A $ \rightarrow $ List A
	rotateRight xs with view xs
	rotateRight $ \lfloor $Nil$ \rfloor $ | Nil = Nil
	rotateRight $ \lfloor $append ys (Cons y Nil)$ \rfloor $ | Snoc ys y = Cons y ys
	\end{lstlisting}
\end{itemize}

\end{frame}

\begin{frame}[fragile]{Building Cryptol's view}
Same steps.
\begin{itemize}
\item Define a data type:
	\begin{lstlisting}[mathescape=true]
	data SplitView {A:Set} : {n:Nat} $ \rightarrow $ (m:Nat) $ \rightarrow $ Vec A (m $\times$ n) $ \rightarrow $ Set where
	$[$_$]$ : forall {m n} $ \rightarrow $ (xss:Vec (Vec A n) m) $ \rightarrow $ SplitView m (concat xss)
	\end{lstlisting}
\item Define a conversion function:
	\begin{lstlisting}[mathescape=true]
	view n m xs = $[$split n m xs$]$
	\end{lstlisting}
\item But this returns \textit{SplitView m (concat (split n m xs))} !
\end{itemize}
\end{frame}

\begin{frame}[fragile]{Building Cryptol's view}
\begin{itemize}
\item We need a lemma:
	\begin{lstlisting}[mathescape=true]
	splitConcatLemma : forall {A n m} $\rightarrow$ (xs:Vec A (m $\times$ n)) $\rightarrow$ concat (split n m xs) $\equiv$ xs
	\end{lstlisting}
\item Conversion function again:
	\begin{lstlisting}[mathescape=true]
	view : {A:Set} $ \rightarrow $ (n:Nat) $ \rightarrow $ (m:Nat) $ \rightarrow $ (xs : Vec A (m $\times$ n)) $ \rightarrow $ SplitView m xs
	view n m xs with concat (split n m xs) | $[$split n m xs$]$  | splitConcatLemma m xs
	view n m xs | $\lfloor$xs$\rfloor$ | v | Refl = v
	\end{lstlisting}
\end{itemize}
\textbf{Note:} Only the view designer has to define all of this.
\end{frame}

\begin{frame}[fragile]{Building Cryptol's view}

\begin{itemize}
\item Finally, we can define the \textit{swab} function:
	\begin{lstlisting}[mathescape=true]
	swab : Word 32 $ \rightarrow $ Word 32
	swab xs with view 8 4 xs
	swab $\lfloor$_$\rfloor$ | $[$a::b::c::d::Nil$]$ 
	= concat $[$b::a::c::d::Nil$]$
	\end{lstlisting}
\end{itemize}

\end{frame}

\begin{frame}[fragile]{Case Study 1 - Discussion}

\begin{itemize}
\item Our Cryptol-like view doesn't take the patterns into account, so we have to explicitly provide this information.
\item Could we define this in Haskell?
	\begin{lstlisting}[mathescape=true]
	data SnocView a = Nil | Snoc (SnocView a) a
	view :: [a] $ \rightarrow $ SnocView a
	\end{lstlisting}
	\begin{itemize}
	\item We lose the link with the original list!
	\item Haskell is too general:
	\begin{lstlisting}[mathescape=true]
	view = const Nil :: [a] $ \rightarrow $ SnocView a
	\end{lstlisting}
	\item Compare with Agda:
	\begin{lstlisting}[mathescape=true]
	view : {A:Set} $ \rightarrow $ (xs:List A) $\rightarrow $ SnocView xs
	\end{lstlisting}
	\end{itemize}
\item Rule of thumb: You can always view data in another way, as long as you \textbf{don't throw away information}.
\end{itemize}

\end{frame}

\subsection{Data Description Languages}
\begin{frame}[fragile]{Data Description Languages - Overview}

\begin{itemize}
\item Data isn't standardized, so we have to write parsers :(
\item Data description languages to the rescue!
\item Give precise description of data format, use it to generate data types and parsers.
\item Unfortunately, still an external tool.
\item Let's implement a simple combinator library in Agda.
\item This case study has a rich generic programming flavour.
\item But first, second DTP concept: \textbf{Universes}.
\end{itemize}

\end{frame}

\begin{frame}[fragile]{DTP concept: Universes}

\begin{itemize}
\item Agda doesn't have type classes! How can we do ad-hoc polymorphism?
\item Type classes are used to describe type collections that support certain operations, like equality.
\item Type theory has the same issue, and we can apply the employed techniques.
\item The type \textit{U} is a collection of 'codes' for types:
	\begin{lstlisting}[mathescape=true]
	data U : Set where
	BIT : U
	CHAR : U
	NAT : U
	VEC : U $ \rightarrow $ Nat $ \rightarrow $ U
	\end{lstlisting}
\end{itemize}

\end{frame}

\begin{frame}[fragile]{DTP concept: Universes}

\begin{itemize}
\item Mapping universe codes to actual types:
	\begin{lstlisting}[mathescape=true]
	el : U $ \rightarrow $ Set
	el BIT = Bit
	el CHAR = Char
	el NAT = Nat
	el (VEC u n) = Vec (el u) n
	\end{lstlisting}
\item A pair of a type \textit{U} and a function \textit{el : U $ \rightarrow $ Set} is a \textbf{universe}.
\textbf{Note:} This is a closed universe.
\item We can define type-generic functions by induction on \textit{U}.
\end{itemize}

\end{frame}

\begin{frame}[fragile]{DTP concept: Universes - Generic functions}
Generic \textit{show}:
	\begin{lstlisting}[mathescape=true]
	show : {u:U} $ \rightarrow $ el u $ \rightarrow $ String
	show {BIT} O = "0"
	show {BIT} I = "1"
	show {CHAR} c = charToString c
	show {NAT} Zero = "Zero"
	show {NAT} (Succ k) = "Succ " $\doubleplus$ parens (show k)
	show {VEC u Zero} Nil = "Nil"
	show {VEC u (Succ k)} (x::xs) = parens (show x) $\doubleplus$ " :: " $\doubleplus$ parens (show xs)
	\end{lstlisting}

\end{frame}

\begin{frame}[fragile]{DDL - Building the Format universe}
	\begin{lstlisting}[mathescape=true]
	data Format : Set where
	Bad : Format
	End : Format
	Base : U $ \rightarrow $ Format
	Plus : Format $ \rightarrow $ Format $ \rightarrow $ Format
	Skip : Format $ \rightarrow $ Format $ \rightarrow $ Format
	Read : (f : Format) $ \rightarrow $ ( $\llbracket$f$\rrbracket$$ \rightarrow $ Format) $ \rightarrow $ Format
	\end{lstlisting}
	\begin{lstlisting}[mathescape=true]
	$\llbracket$_$\rrbracket$ : Format $ \rightarrow $ Set
	$\llbracket$Bad$\rrbracket$ = Empty
	$\llbracket$End$\rrbracket$ = Unit
	$\llbracket$Base u$\rrbracket$ = el u
	$\llbracket$Plus f1 f2$\rrbracket$ Either $\llbracket$f1$\rrbracket$ $\llbracket$f2$\rrbracket$
	$\llbracket$Read f1 f2$\rrbracket$ Sigma $\llbracket$f1$\rrbracket$ ($\lambda$x $ \rightarrow $ $\llbracket$f2$\rrbracket$ x)
	$\llbracket$Skip _ f$\rrbracket$ = $\llbracket$f$\rrbracket$
	\end{lstlisting}
\end{frame}

\begin{frame}[fragile]{DDL - Format combinators}
	\begin{lstlisting}[mathescape=true]
	char : Char $ \rightarrow $ Format
	char c = Read (Base CHAR) ($\lambda$c' $ \rightarrow $ if c $ \equiv $ c' then End else Bad)

	satisfy : (f:Format) $ \rightarrow $ ($\llbracket$f$\rrbracket$ $ \rightarrow $ Bool) $ \rightarrow $ Format
	satisfy f pred = Read f ($\lambda$x $ \rightarrow $ if (pred x) then End else Bad)

	_ $ \gg $ _ : Format $ \rightarrow $ Format $ \rightarrow $ Format
	f1 $ \gg $ f2 = Skip f1 f2

	_ $ \monadbind $ _ : (f : Format) $ \rightarrow $ ($\llbracket$f$\rrbracket$ $ \rightarrow $ Format) $ \rightarrow $ Format
	x $ \monadbind $ f = Read x f
	\end{lstlisting}
\end{frame}

\begin{frame}[fragile]{Example Format}
The NETPBM format:\newline
\texttt{P4 100 60}\newline
\texttt{OIOOOOOOOOOIIIOIIIIOOIIIIIIIIOOO...}

	\begin{lstlisting}[mathescape=true]
	pbm : Format
	pbm = char 'P' $ \gg $ char '4' $ \gg $ char ' ' $ \gg $
	      Base NAT $ \monadbind $ $\lambda$n $ \rightarrow $ char ' ' $ \gg $
	      Base NAT $ \monadbind $ $\lambda$m $ \rightarrow $ char '\n' $ \gg $
	      Base (VEC (VEC BIT m) n) $ \monadbind $ $ \lambda $bs $ \rightarrow $ End
	\end{lstlisting}
\end{frame}

\begin{frame}[fragile]{DDL - Generic parsers}
	\begin{lstlisting}[mathescape=true]
	parse : (f:Format)$ \rightarrow $List Bit$ \rightarrow $Maybe ($\llbracket$f$\rrbracket$,List Bit)
	parse Bad bs = Nothing
	parse End bs = Just (unit, bs)
	parse (Base u) bs = read u bs
	parse (Plus f1 f2) bs with parse f1 bs
	... | Just (x,cs) = Just (Inl x, cs)
	... | Nothing with parse f2 bs
	... | Just (y, ds) = Just (Inr y, ds)
	... | Nothing = Nothing 
	parse (Skip f1 f2) bs with parse f1 bs
	... | Nothing = Nothing
	... | Just (_,cs) = parse f2 cs
	parse (Read f1 f2) bs with parse f1 bs
	... | Nothing = Nothing
	... | Just (x,cs) with parse (f2 x) cs
	... | Nothing = Nothing
	... | Just (y,ds) = Just (Pair x y, ds) 
	\end{lstlisting}
\end{frame}

\begin{frame}[fragile]{DDL - Generic printers}
	\begin{lstlisting}[mathescape=true]
	print : (f:Format)$ \rightarrow $  $\llbracket$f$\rrbracket$ $ \rightarrow $ List Bit
	print Bad ()
	print End _ = Nil
	print (Base u) x = toBits (show x)
	print (Plus f1 f2) (Inl x) = print f1 x
	print (Plus f1 f2) (Inr x) = print f2 x
	print (Read f1 f2) (Pair x y) 
	  = append (print f1 x) (print (f2 x) y)
	print (Skip f1 f2) = ???
	\end{lstlisting}
\end{frame}

\begin{frame}[fragile]{DDL - Generic printers - Fixing Skip}

\begin{itemize}
\item To print \textit{Skip f1 f2}, we need values of types $\llbracket$f1$\rrbracket$ and $\llbracket$f2$\rrbracket$, but we only have $\llbracket$f2$\rrbracket$!
\item Solution: change the type of \textit{Skip}: 
\begin{lstlisting}[mathescape=true]
	Skip : (f:Format) $\rightarrow$ $\llbracket$f$\rrbracket$ $\rightarrow$ Format $\rightarrow$ Format
\end{lstlisting}
\item The new value of type $\llbracket$f$\rrbracket$ can be used to print out the \textit{f1}.
\item Update the \textit{print} function:
	\begin{lstlisting}[mathescape=true]
	print : (f:Format)$ \rightarrow $  $\llbracket$f$\rrbracket$ $ \rightarrow $ List Bit
	print (Skip f1 v f2) x 
	  = append (print f1 v) (print f2 x)
	\end{lstlisting}
\end{itemize}

\end{frame}

\begin{frame}[fragile]{Case Study 2 - Discussion}

\begin{itemize}
\item Our \textit{Format} data type does not support recursion: Agda complains of possible non-termination.
\item Possible solutions:
	\begin{itemize}
	\item Extend \textit{Format} with another constructor: \textit{Many : Format $ \rightarrow $ Format}
	\item More generally: Extend it with variables and least-fixed points.
	\end{itemize}
\item A lot like Haskell-like generic programming. However:
	\begin{itemize}
	\item Agda uses dependent pairs, while Haskell uses normal pairs. In Parsec, we can parse a "Vector":
	\begin{lstlisting}[mathescape=true]
	parseVec = do n $ \leftarrow $ parseInt
	              xs $ \leftarrow $ count n parseBit
	              return (n,xs)
	\end{lstlisting}
	\item Returns a (Int,[Bit]) : The link between both elements is lost!
	\end{itemize}
\item $\llbracket$f$\rrbracket$ is usually a nested tuple of values. But we can define a record-like view if we want.
\end{itemize}
Metatheoretical note: We can get a value of type $\llbracket$f$\rrbracket$ only if we can construct one (which implies succesful parsing). Hence, we get this important metatheoretic property (receiving a value of $\llbracket$f$\rrbracket$ $\Leftrightarrow$ succesful parse) for free!

\end{frame}

\subsection{Relational Algebra}
\begin{frame}[fragile]{Relational Algebra - Overview}

\begin{itemize}
\item Database communication is a crucial element in modern computing. However, some issues:
	\begin{itemize}
	\item Hardly any static checking: easy to make queries that make no sense.
	\item Programmers have to learn (and switch to) another language for communication.
	\end{itemize}
\item Several EDSLs for database queries in Haskell available. Again, several issues:
	\begin{itemize}
	\item Problematic to express all concepts of relational algebra (especially join and cartesian product)
	\item Several language extensions required (MultiParamTypeClasses, Extensible Records, Fundeps)
	\item Have to know what kind of data a database contains. Usually in the form of a preprocessor.
	\end{itemize}
\item Because of these issues, many libraries resort to dynamic typing.
\item The root of the problem? Haskell's type system differs fundamentally from DB query language.
\item We can do better with Agda!
\end{itemize}

\end{frame}

\begin{frame}[fragile]{Relational Algebra - Schemas, tables, rows}
An example table: \newline
    \begin{tabular}{ | l | l | l |}
    \hline
    Model & Time & Wet \\ \hline
    Ascari A10 & 1:17.3 & False \\ \hline
    Koenigsegg CCX & 1:17.6 & True\\ \hline
    Pagani Zonda C12 F & 1:18.4 & False \\ \hline
    Maserati MC12 & 1:18.9 & False \\ \hline
    \end{tabular}

\begin{itemize}
\item \textbf{Schema:} Type of a table:
\begin{lstlisting}[mathescape=true]
	Schema : Set
	Schema = List Attribute
	Attribute : Set
	Attribute = (String, U)
\end{lstlisting}
\item Example schema for our example table:
\begin{lstlisting}[mathescape=true]
	Cars : Schema
	Cars = Cons ("Model", VEC CHAR 20) (Cons ("Time", VEC CHAR 6) (Cons ("Wet", BOOL) Nil))
\end{lstlisting}
\end{itemize}

\end{frame}

\begin{frame}[fragile]{Relational Algebra - Schemas, tables, rows}

\begin{itemize}
\item We can then define tables as lists of rows:
\begin{lstlisting}[mathescape=true]
	data Row : Schema $\rightarrow$ Set where
	EmptyRow : Row Nil
	ConsRow : forall {name u s} $ \rightarrow $ el u $ \rightarrow $ Row s $\rightarrow$ Row (Cons (name,u) s)

	Table : Schema $\rightarrow$ Set
	Table s = List (Row s)
\end{lstlisting}
\item Example row of our table:
\begin{lstlisting}[mathescape=true]
	zonda : Row Cars
	zonda = ConsRow "Pagani Zonda C12 F" (ConsRow "1:18.4" (ConsRow False EmptyRow))
\end{lstlisting}
\item Heterogenous lists! More complex in Haskell.
\end{itemize}

\end{frame}

\begin{frame}[fragile]{Relational Algebra - Setting up a connection}

\begin{itemize}
\item Most Haskell database interfaces provide functions like these:
\begin{lstlisting}[mathescape=true]
	connect :: ServerName $\rightarrow$ IO Connection
	query :: String $\rightarrow$ Connection $\rightarrow$ IO String
\end{lstlisting}
\item Types are very poor: no static checks possible.
\item With dependent types, we can be far more precise:
\begin{lstlisting}[mathescape=true]
	Handle : Schema $\rightarrow$ Set
	connect : ServerName $\rightarrow$ TableName $\rightarrow$ (s:Schema) $\rightarrow$ IO (Handle s)
\end{lstlisting}
\item We connect to a specific table of the database.
\item We even provide the schema to which the table should adhere to!
\item \textit{connect} function asks DB for a table description, parses it and compares it to \textit{s}.
\item If connection succeeds, the rest of the program cannot go wrong.
\end{itemize}

\end{frame}

\begin{frame}[fragile]{Relational Algebra - Constructing queries}
Let's embed relational algebra operators in Agda:
\begin{lstlisting}[mathescape=true]
	RA : Schema $\rightarrow$ Set whre
	Read : forall {s} $\rightarrow$ Handle s $\rightarrow$ RA s
	Union : forall {s} $\rightarrow$ RA s $\rightarrow$ RA s $\rightarrow$ RA s
	Diff : forall {s} $\rightarrow$ RA s $\rightarrow$ RA s $\rightarrow$ RA s
	Product : forall {s s'} $\rightarrow$ {So (disjoint s s')} $\rightarrow$ RA s $\rightarrow$ RA s' $\rightarrow$ RA (append s s')
	Project : forall {s} $\rightarrow$ (s' : Schema) $\rightarrow$ {So (sub s' s)} $\rightarrow$ RA s $\rightarrow$ RA s'
	Select : forall {s} $\rightarrow$ SQLExpr s BOOL $\rightarrow$ RA s $\rightarrow$ RA s
\end{lstlisting}

\begin{lstlisting}[mathescape=true]
	So : Bool $\rightarrow$ Set
	So True = Unit
	So False = Empty
\end{lstlisting}

\end{frame}

\begin{frame}[fragile]{Relational Algebra - Constructing queries}
\begin{itemize}
\item Project example:
\end{itemize}
\begin{lstlisting}[mathescape=true]
	Models : Schema
	Models = Cons ("Model", VEC CHAR 20) Nil
	models : Handle Cars $\rightarrow$ RA Models
	models h = Project Models (Read h)
\end{lstlisting}
\begin{itemize}
\item Select needs a way to filter results:
\end{itemize}
\begin{lstlisting}[mathescape=true]
	data SQLExpr : Schema $\rightarrow$ U $\rightarrow$ Set where
	equal : forall {u s} $\rightarrow$ SQLExpr s u $\rightarrow$ SQLExpr s u $\rightarrow$ SQLExpr s BOOL
	lessThan : forall {u s} $\rightarrow$ SQLExpr s u $\rightarrow$ SQLExpr s u $\rightarrow$ SQLExpr s BOOL
	_!_ : (s:Schema) $\rightarrow$ (name:String) $\rightarrow$ {So (occurs name s)} $\rightarrow$ SQLExpr s (lookup name s p)
\end{lstlisting}

\begin{lstlisting}[mathescape=true]
	wet : Handle Cars $\rightarrow$ RA Models
	wet h = Project Models (Select (Cars ! "Wet") (Read h))
\end{lstlisting}

\end{frame}

\begin{frame}[fragile]{Relational Algebra - Executing queries}

\begin{itemize}
\item We know how to construct queries, but how can we send them?
\item Naive approach:
\begin{lstlisting}[mathescape=true]
	toSQL: forall {s} $\rightarrow$ RA s $\rightarrow$ String
\end{lstlisting}
\item We lose a lot of type information! Better approach:
\begin{lstlisting}[mathescape=true]
	query : {s:Schema} $\rightarrow$ RA s $\rightarrow$ IO (List (Row s))
\end{lstlisting}
\item We now know how to parse the DB's response in a type-safe way.
\end{itemize}

\end{frame}

\begin{frame}[fragile]{Case Study 3 - Discussion}

\begin{itemize}
\item \textit{Schema} is a List, so there can be duplicates, and element order matters.
\item Modify \textit{Cons} constructor for no duplicates:
\begin{lstlisting}[mathescape=true]
	Cons : (name:String) $\rightarrow$ (u:U) $\rightarrow$ (s:Schema) $\rightarrow$ {So ($\neg$(elem name s))} $\rightarrow$ Schema
\end{lstlisting}
\item Making element order irrelevant is harder.
	\begin{itemize}
	\item Quotient types (to hide information)?
	\item Add proof arguments to our constructors?
	\begin{lstlisting}[mathescape=true]
	Union : forall {s s'} $\rightarrow$ {So (permute s s')} $\rightarrow$ RA s $\rightarrow$ RA s' $\rightarrow$ RA s
	\end{lstlisting}
	\item Use sorted list or trie?
	\end{itemize}
\item Lap time was modeled as fixed-length string, why not use a triple of integers?
	\begin{itemize}
	\item DB's only support a limited amount of datatypes
	\item Using views, we can marshall data to and fro
	\end{itemize}
\end{itemize}

\end{frame}

\section{Conclusions}
\begin{frame}[fragile]{Conclusions}

\begin{itemize}
\item Unlike Haskell, we can compute new types from data:
	\begin{itemize}
	\item File format description $\rightarrow$ compute type of the parser
	\item Compute type of a table given a description
	\end{itemize}
\item We can have precise data types in dependently-typed languages. (The head of an empty list is an absurd value, but it is possible in Haskell!)
\item With views, it is possible to destruct data in a custom manner. (Haskell struggles to offer such support.)
\item Generic programming is a hot topic in Haskell right now. Universes and its assorted techniques can be implemented even more elegantly in a dependently-typed language.
\item There are many papers about type systems being published to solve specific problems. With a dependently-typed language, we can experiment with types as much as we want, and \textbf{spend our time writing programs instead of typing rules}.
\end{itemize}

\end{frame}

\begin{frame}[fragile]

\begin{center}
Thanks for your time! Any questions?
\end{center}

\end{frame}


\end{document}
