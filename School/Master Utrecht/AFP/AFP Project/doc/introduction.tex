The development of dynamic web pages is a field that has been mostly dominated by dynamic programming languages. % \cite{}
These languages tend to have a good level of expressibility, flexibility and they usually provide good (or even just ``good enough'') performance for this use.
However, they usually lack most of the advantages many functional programming languages provide, such as type--safety and referential transparency.
The use of the web has also evolved in the past decade to provide sets of dynamic web pages as a single application for end--users, following a client/server model.
Given the stateless nature of HTTP as the underlying protocol this only aggravates the need for a set of good programming tools to abstract away from this limitation.
In this report we present an elaboration on this issue and some solutions developed using the Clean programming language and some web frameworks for Haskell.

Haskell is a functional programming language that has been presented by some as an option for web development. \cite{snap,warp}
It is a lazy and strongly typed language that provides the programmer with type safety that can catch several common programming mistakes.
However, the presence of a type inference system also frees the programmer from having to extensively and explicitly declare types, reducing the amount of work required, and giving the same ``feel'' provided by dynamic languages such as Ruby or Python, which are popular for web development.
The availability of a mature implementation in the form of the Haskell Platform also enables the language user to take advantage of the language laziness without taking too much or any of a performance hit.

From a programming point of view, the major issue with interactive applications based on the web is the stateless nature of the environment.
Most developed solutions to this issue involve some sort of abstraction mechanism by storing and reloading state information on each connection.
The iTask work flow management framework developed in Clean presents an alternative solution to solving the statelessness issue by storing partial computation as the state data.
\sref{problem} presents the issue and these developments with more detail.

In general, a web framework will provide state management and I/O both from web connections and from back--end storages such as databases.
Several frameworks provide other features such as automatic URL generation, type--safe HTML form generation, extensive authentication capabilities and in--memory databases.
Furthermore, some frameworks provide scaffolding tools for easy installation and configuration of a website.
In \sref{frameworks} we present the following frameworks for web development in Haskell:
\begin{enumerate}
	\item Happstack
	\item Snap
	\item Yesod
\end{enumerate}
These frameworks have been selected because of our impression that they are the most promising frameworks with the most active development communities.
In \sref{comp}, we present a more in--depth comparison of the advanced features that are present in the libraries available for the Haskell developer.
We also present a comparison of the web frameworks cited above considering both the advanced features and external factors such as quality of documentation and project status.

In section \ref{conclusion} we present a conclusion of the comparison of frameworks and list some of the promising developments.