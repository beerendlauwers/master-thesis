\documentclass[10pt,usenames,dvipsnames]{beamer}
\usepackage[utf8]{inputenc}
\usepackage{amsmath}
\usepackage{amsfonts}
\usepackage{amssymb}
\usepackage{listings}
\usepackage{natbib}
\usepackage{color}
\usetheme{Antibes}
\lstset{language=Haskell}
\lstset{breaklines=true}
\lstset{basicstyle=\scriptsize}

\newcommand{\newblock}{}
\newcommand{\sufficient}{\tiny{$ \square $}}

\author{
	Beerend Lauwers\\
	\and 
	Augusto Martins\\	
	\and
	Frank Wijmans\newline\\
	\hbox{\{B.Lauwers, A.PassalaquaMartins, F.S.Wijmans\}@students.uu.nl}
	\and
	\newline Utrecht University, The Netherlands}
\date{\today}

\title{Comparing Haskell Web Frameworks}

\begin{document}

	\frame{\titlepage}

	\begin{frame}{Outline}
		\tableofcontents
	\end{frame}
	
	\section{Introduction}
	
	\subsection*{Motivation}
	
	\begin{frame}{Motivation}
		Web applications are everywhere.
		Advantages for using the web approach for your application:
		\begin{itemize}
			\item No installing
			\item Thin client
			\item Updates without required user actions
			\item Easy integration
			\item Platform free; needs web browser
		\end{itemize}
	\end{frame}
	
	\subsection*{Problem}
	
	\begin{frame}{Problem}
		Inherent problems of the web paradigm.
		\begin{itemize}
			\item HTTP Protocol
			\begin{itemize}
			\item Stateless nature
			\end{itemize}
			\item User input
			\begin{itemize}
			\item Untyped
			\item Security problems
			\end{itemize}
			\item Error handling
			\item Servers are single point of failure
			\begin{itemize}
			\item Denial-of-service
			\end{itemize}
		\end{itemize}
	\end{frame}
	
	\subsection*{Clean: iData}
	
	\begin{frame}{Clean: iData}
		\begin{itemize}
			\item Clean, developed in Nijmegen.
			\item It allows saving partial computations, where it is different from Haskell.
			\item iData (for \textit{i}nteractive Data) elements
			\begin{itemize}
				\item Conversion HTML forms.
				\item These elements can be saved on server, in html form or in session.
			\end{itemize}
			\item Generic programming
			\begin{itemize}
				\item Mapping types to forms and user input to types.
			\end{itemize} 
		\end{itemize}
	\end{frame}
	
	\subsection*{Haskell solution}
	
	\begin{frame}{Haskell solution}
		Haskell brings some unique language features to the table, compared to imperative languages.
		\begin{itemize}	
			\item \textbf{Type safe} user input and database connections, eliminating security issues from the user.
			\item Usually, websites depend on other languages: JavaScript, HTML/XML, CSS and SQL.
		\end{itemize}
	\end{frame}

	\begin{frame}{Web frameworks}
		Many frameworks cover (some subset of) the features needed for web applications.
		
		\begin{enumerate}
			\item \textbf{Happstack}
			\item \textbf{Snap}
			\item \textbf{Yesod}
			\item Haskell on a Horse
			\item miku
			\item Lemmachine
			\item mohws
			\item Salvia (Sebastiaan Visser, Utrecht University)
			\item Scotty 
		\end{enumerate}
		
		Lets see how the top three differ, and how they score on the different features.
	\end{frame}
	
	%
	% This is where Beerend takes over.
	%
	
	\section{Features}
	
	\begin{frame}{Feature analysis}
		A web framework ought to provide several functionalities.
		Let's walk through them from our common point of view, that of a functional programmer.
	\end{frame}
	
	\subsection*{Path routing}
	\begin{frame}{Path routing}
		URL's can be static or dynamic. (= query string element)
		
		Ideally, when the user requests a path the server returns:
		\begin{itemize}
			\item The user-requested page
			\item An error page (i.e. no credentials or incorrect user action)
		\end{itemize} 
		
		$ \rightarrow $ Use Haskell's type safety to route a request and generate its appropriate response
	\end{frame}
	
	\begin{frame}
		\begin{itemize}
		\item \textbf{Happstack}: Concept of $ guards $
		\begin{itemize}
			\item $ dir $ and $ dirs $ ($ ServerMonad / MonadPlus $)
			\item Extract part of URL ($ FromReqUri $ class)
			\item Request method ($ MatchMethod $ class)
			\item Arbitrary function ($ guardRq $)
			\item Not fully type--safe
		\end{itemize}
		\item\textbf{Snap}: Sum of its parts
		\begin{itemize}
			\item Each Snaplet generates routes
			\item Routes can point to sub--Snaplets / directories
			\item Snap core unifies all routes
			\item Validity of routes must be confirmed by programmer
		\end{itemize}
		\item \textbf{Yesod}: built--in type--safe routing
		\begin{itemize}
			\item Split URL at forward slashes
			\item Attempt to match with sitemap
			\item Look up proper $ Handler $ function
			\item ``short--circuiting'' behaviour: escape a computation ($ MEither $)
		\end{itemize}
		\end{itemize}
	\end{frame}

	\begin{frame}[fragile]{Path routing: Yesod example}
		Sitemap:
		\begin{lstlisting}[mathescape=true]
		/user/$ \textcolor{Red}{\# String} $	$ \textcolor{Green}{UserR} $		GET POST
		/faq	        $ \textcolor{Green}{FaqR} $
		\end{lstlisting}
		Sitemap datatype:
		\begin{lstlisting}[mathescape=true]
		data Route = $ \textcolor{Green}{UserR} $ $ \textcolor{Red}{String} $ | $ \textcolor{Green}{FaqR} $
		\end{lstlisting}
		Handlers:
		\begin{lstlisting}[mathescape=true]
		handle$ \textcolor{Green}{FaqR} $ :: Handler RepHtml
		get$ \textcolor{Green}{UserR} $, post$ \textcolor{Green}{UserR} $ :: $ \textcolor{Red}{Text} $ $ \rightarrow $ Handler RepHtml
		\end{lstlisting}

	\end{frame}

	\begin{frame}
		\begin{itemize}
		\item \textbf{web--routes}: external library
		\begin{itemize}
			\item Map URL types to Strings and back
			\item Highly flexible
			\item Define datatype (e.g. $ SiteMap $) representing all valid routes
			\item $ RouteT $ monad transformer wraps around web server monad
			\item Guards against incorrect URL's: $ RouteT $ $ SiteMap $
			\item route :: SiteMap $ \rightarrow $ RouteT SiteMap (ServerPartT IO) Response
		\end{itemize}
		\item \textbf{web--routes--boomerang}: extends \textbf{web--routes}
		\begin{itemize}
			\item Single grammar for parsing and printing
			\item Greater control over URL appearance while retaining maintainability
		\end{itemize}
		\end{itemize}
	\end{frame}
	
	\subsection*{Serving static files}
	
	\begin{frame}{Serving static files}
		\begin{itemize}
		\item \textbf{Happstack}: Highly granular control
		\begin{itemize}
			\item Manipulate $ Response $ types ($ FilterMonad $)
			\item Escape current computation and return different $ Response $ ($ WebMonad $)
			\item $ serveDirectory $ and $ serveFile $ + guards from previous section
			\item Roll your own functions with $ Happstack.Server.FileServe.BuildingBlocks $
		\end{itemize}
		\item\textbf{Snap}: Utility module
		\begin{itemize}
			\item $ Snap.Util.FileServe $
			\item Automatic / Customized generation of directory indices
			\item Dynamic MIME--type handlers (e.g. pretty--printing source code)
		\end{itemize}
		\item \textbf{Yesod}: Minimal functionality, leaves static file serving to server implementation
		\end{itemize}
	\end{frame}	
	
	\subsection*{HTML generation}
	
	\begin{frame}{HTML generation}
		\begin{itemize}
		\item \textbf{Happstack}: \textbf{BlazeHtml}
		\item\textbf{Snap}: \textbf{Heist} (originally in--house)
		\item \textbf{Yesod}: \textbf{Hamlet} (originally in--house)
		\end{itemize}

		\begin{itemize}
		\item \textbf{BlazeHtml}: Fast, combinator--based, compiled
		\item\textbf{Heist}: XML templating engine, runtime
		\item \textbf{Hamlet}: Quasi--quotation, type--safe, compiled
		\item \textbf{HSP}: Embedded XML, compiled
		\item \textbf{HStringTemplate}: Based on Java's StringTemplate library, compiled / runtime
		\end{itemize}

		\begin{itemize}
		\item \textbf{happstack--plugins}: Automatic type--checking, recompilation and reloading of modules on a running server
		\end{itemize}
	\end{frame}
	
	\subsection*{JavaScript and CSS Generation}
	
	\begin{frame}{JavaScript and CSS Generation}
		\begin{itemize}
		\item \textbf{Happstack}: \textbf{JMacro} (JavaScript)
		\item\textbf{Snap}: None
		\item \textbf{Yesod}: $ Cassius $ (CSS), $ Lucius $ (CSS) and $ Julius $ (JavaScript) quasi-quoters
		\begin{itemize}
			\item $ Cassius $: Insertion of variables and URL's, whitespace mark--up
			\item $ Lucius $: Insertion of variables and URL's, standard CSS mark--up, CSS nesting
			\item $ Julius $: Insertion of variables, URL's, and templates
		\end{itemize}
		\end{itemize}

		\begin{itemize}
		\item \textbf{JMacro}: external library
		\begin{itemize}
			\item Haskell values and --techniques
			\begin{itemize}
				\item Lambda expressions: fun addTwo x $ \rightarrow \backslash $y $ \rightarrow $ x + y;  
				\item Haskell--style function application: var multThenAdd = add 2 (mult 2 3);
			\end{itemize}
			\item Automatic variable scoping: prevent overlap with variables of the same name
			\item Antiquotation: use Haskell code directly within JavaScript code
		\end{itemize}
		\end{itemize}
	\end{frame}	
	
	\subsection*{Parsing request data and form generation}
	
	\begin{frame}{Parsing request data and form generation}
		\begin{itemize}
		\item \textbf{Happstack}
		\begin{itemize}
			\item  \textbf{Parsing request data}: Built--in
			\begin{itemize}
				\item Extract from query string, GET and POST, cookies ($ HasRqData $ monad)
				\item $ look $, $ lookCookie $, $ lookRead $
				\item Custom parsing and error messages with $ checkRq $
				\item Optional parameters: $ optional $ function from $ Control.Applicative $
			\end{itemize}
			\item  \textbf{Form generation}: \textbf{digestive--functors} library
		\end{itemize}
		\item\textbf{Snap}
		\begin{itemize}
			\item  \textbf{Parsing request data}: Built--in
			\begin{itemize}
				\item $ rqCookies $, $ rqQueryString $, $ rqParams $
				\item Modify parameters: $ rqModifyParams  $, $ rqSetParam $
			\end{itemize}
			\item  \textbf{Form generation}: \textbf{digestive--functors} library
		\end{itemize}
		\item \textbf{Yesod}: Both parsing and form generation in \textbf{yesod--form}
		\begin{itemize}
				\item Type--safe form generation
				\item Form parsing into Haskell datatypes
				\item JavaScript validation code generation
				\item Applicative way of combining forms similar to \textbf{digestive--functors}
			\end{itemize}
		\end{itemize}
	\end{frame}	

	\begin{frame}[fragile]
		\begin{itemize}
		\item \textbf{digestive--functors}: external library
		\begin{itemize}
			\item Inspired by \textbf{Formlets} library
			\item Combine simple form elements into more complex forms
			\item $ FormState $ newtype: $ State $ monad that provides unique identifiers to each $ <input> $ tag
		\end{itemize}
		\end{itemize}


		\begin{lstlisting}
		data Info = Person String String Int deriving (Show)
		infoForm = User <$> inputText Nothing <*> inputText Nothing <*> inputTextRead "Could not parse value" (Just 25)
		\end{lstlisting}

		\begin{lstlisting}
		data Group = Persons Info Info Info deriving (Show)
		groupForm = Persons <$> infoForm <*> infoForm <*> infoForm
		\end{lstlisting}

	\end{frame}
	
	\subsection*{Sessions and state handling}
	
	\begin{frame}{Sessions and state handling}
		\begin{itemize}
		\item \textbf{Happstack}: \textbf{happstack--auth} and \textbf{happstack--extra}
		\begin{itemize}
			\item Both external libraries
			\item Both simple session management
			\item \textbf{happstack--extra}: Session ID saved in cookie, rest in--memory on server
		\end{itemize}
		\item\textbf{Snap}: \textbf{snap-auth} and \textbf{mysnapsession}
		\begin{itemize}
			\item  \textbf{snap-auth}
			\begin{itemize}
				\item Uses Yesod's \textbf{clientsession} as a base
				\item Adds ability to save extra data in session cookie
			\end{itemize}
			\item  \textbf{mysnapsession}: external library
			\begin{itemize}
				\item Simple session management, supports in--memory sessions
				\item Continuation--based programming model for multiple--request stateful interactions
			\end{itemize}
		\end{itemize}
		\item \textbf{Yesod}: \textbf{clientsession} (originally in--house)
		\begin{itemize}
			\item Session data is stored in cookie
			\item Message system
			\item Integrates with \textbf{yesod--auth}'s ultimate destination design
		\end{itemize}
		\end{itemize}		
	\end{frame}	
	
	\subsection*{Persistence}
	
	\begin{frame}{Persistence}
		
		\begin{itemize}
		\item \textbf{Happstack}: \textbf{acid--state} (originally in--house)
		\begin{itemize}
			\item In--Memory ACID data store, non--relational
			\item Can store arbitrary Haskell values
			\item Write to file functions along with parameters
			\item Restoring state = re--running logs
			\item Uses \textbf{SafeCopy} for data migration and extension
			\item Type--safe querying and updating
		\end{itemize}
		\item\textbf{Snap}: Snaplets for \textbf{HDBC}, \textbf{mongoDB} and \textbf{acid--state}
		\item \textbf{Yesod}: \textbf{persistent} (originally in--house)
		\begin{itemize}
			\item Define datatype, derive database-explainable version using TH
			\item Can store arbitrary Haskell values
			\item Databases supported out of the box: MongoDB, PostgreSQL, SQLite
			\item Support for data migration and extension, dry runs
			\item Generated primary key datatype enforces type--safe relational queries
			\item Type--safe insert, update, delete queries written in Haskell:
			\item persons $ \leftarrow $ selectList [Age ==. 22] [Asc Age]
			\item Uniqueness constraints, default values, nullability, foreign keys
		\end{itemize}
		\end{itemize}
		
	\end{frame}

	\begin{frame}[fragile]{Persistence: acid--state example}
		Create datatype:
		\begin{lstlisting}
		data CounterState = CounterState { count :: Integer } deriving (Eq, Ord, Read, Show, Data, Typeable)
		$(deriveSafeCopy 0 'base ''CounterState)
		\end{lstlisting}
		Update function:
		\begin{lstlisting}[mathescape=true]
		incCountBy :: Integer $ \rightarrow $ Update CounterState Integer
		incCountBy n =
		 do c@CounterState{..}  $ \leftarrow $ get -- uses RecordWildCards extension
		 let newCount = count + n
		 put $ \$ $ c { count = newCount }
		 return newCount
		\end{lstlisting}
		Query function:
		\begin{lstlisting}
		peekCount :: Query CounterState Integer
		peekCount = count <$> ask
		\end{lstlisting}
		Ensure ACIDity:
		\begin{lstlisting}
		$(makeAcidic ''CounterState ['incCountBy, 'peekCount])
		\end{lstlisting}
	\end{frame}

	\begin{frame}[fragile]
		$ makeAcidic $ generates the following:
		\begin{lstlisting}[mathescape=true]
		data PeekCount  = PeekCount
		data IncCountBy = IncCountBy Integer
		instance IsAcidic CounterState where
		 acidEvents = [ UpdateEvent (\(IncCountBy newState) $ \rightarrow $ incCountBy newState), QueryEvent (\PeekCount $ \rightarrow $ peekCount) ]
		instance Method PeekCount where
		 type MethodResult PeekCount = Integer
		 type MethodState PeekCount = CounterState
		instance QueryEvent PeekCount
		instance Method IncCountBy where
		 type MethodResult IncCountBy = ()
		 type MethodState IncCountBy = CounterState
		instance UpdateEvent IncCountBy
		-- And more
		\end{lstlisting}
		Querying and updating:
		\begin{lstlisting}[mathescape=true]
		-- P.S.: type EventState ev = MethodState ev
		--       type EventResult ev = MethodResult ev
		update' :: (UpdateEvent event, MonadIO m) $ \Rightarrow $ AcidState (EventState event) $ \rightarrow $ event $ \rightarrow $ m (EventResult event)
		query'  :: (QueryEvent event , MonadIO m) $ \Rightarrow $ AcidState (EventState event) $ \rightarrow $ event $ \rightarrow $ m (EventResult event)
		\end{lstlisting}
		Example update:
		\begin{lstlisting}
		c <- update' acid (IncCountBy 1)
		\end{lstlisting}
	\end{frame}	

	\begin{frame}[fragile]{Persistence: persistent example}
	\fboxsep=1pt
	\fboxrule=1pt
	Generate DB--explainable code:
\begin{minipage}[t]{0.6\textwidth}
		\begin{lstlisting}[mathescape=true]
		mkPersist sqlSettings [persist|
		$ \fcolorbox{Blue}{White}{\parbox{100\unitlength}{
		Person \\[2pt]
		  \fcolorbox{Red}{White} {name String} \\[2pt]
		  \fcolorbox{Red}{White} {age Int} }} $ |]
		\end{lstlisting}
\end{minipage}
\begin{minipage}[t]{0.3\textwidth}
\textcolor{Blue}{PersistEntity}
\fcolorbox{Blue}{White}{\textcolor{Red}{PersistField} \linebreak
\fcolorbox{Red}{White}{\parbox{60\unitlength}{PersistValue}}
}
\end{minipage}
	Example query:
	\begin{lstlisting}[mathescape=true]
	persons $ \leftarrow $ selectList [Age ==. 22] [Asc Age, LimitTo 10]
	\end{lstlisting}
	Type--safe foreign keys:
	\begin{lstlisting}[mathescape=true]
	Car
	 year Int
	 owner PersonId
	\end{lstlisting}

	\end{frame}	
	
	\subsection*{Authentication}
	
	\begin{frame}{Authentication}
		\begin{itemize}
		\item \textbf{Happstack}: \textbf{happstack-authenticate} (based upon \textbf{authenticate})
			\begin{itemize}
			\item \textbf{authenticate} features
			\item Multiple authentication methods per account
			\item Multiple personalities per user account
			\item themeable \textbf{BlazeHtml}--based templates
			\end{itemize}
		\item\textbf{Snap}: \textbf{snap--auth}
			\begin{itemize}
			\item Basic user/password authentication functionality
			\end{itemize}
		\item \textbf{Yesod}: \textbf{yesod--auth} (based upon \textbf{authenticate}, originally in--house)
			\begin{itemize}
			\item \textbf{authenticate} features
			\item E--mail, Google Mail (OpenID), HashDB
			\end{itemize}
		\end{itemize}
		\begin{itemize}
		\item \textbf{authenticate}: external library
			\begin{itemize}
			\item BrowserId, Facebook Connect, Kerberos, OAuth, OpenID and rpxnow
			\end{itemize}
		\end{itemize}
	\end{frame}

	\subsection*{Project management}

	\begin{frame}{Project management}
		\begin{itemize}
		\item \textbf{Happstack}: No tools
		\item\textbf{Snap}: Some commands
		\begin{itemize}
			\item Bare--bones skeleton
			\item Simple, fully--working ``Hello World!'' example
		\end{itemize}
		\item \textbf{Yesod}: Scaffolding tool
		\begin{itemize}
			\item Asks user some questions
			\item Generates cabal package with skeleton Yesod project
		\end{itemize}
		\end{itemize}
	\end{frame}
	
	%
	% This is for Augusto.
	%
	
	\section{Web frameworks}
		
	\subsection{Happstack}
	
	\begin{frame}{Happstack}
			
		Advantages:
		\begin{itemize}
			\item Flexible and complete server
			\item Using alternative packages is possible
			\item ``Crash course'' helps beginners
			\item Framework with most features
		\end{itemize}
		\pause
		Disadvantages:	
		\begin{itemize}
			\item Not the fastest server
			\item Might be too extensive for simple applications
			\item Lacks a project management tool
		\end{itemize}
	\end{frame}
	
	\subsection{Snap}
	
	\begin{frame}{Snap}
			
		Advantages:
		\begin{itemize}
			\item Snaplets, which are easy to use
			\item Standalone libraries are easily integrated
			\item Well documented
			\item Good for smaller applications or applications with different module structure
			\item Scaffolding tool for management
		\end{itemize}
		\pause
		Disadvantages:	
		\begin{itemize}
			\item Advanced features require external libraries
			\item Only a mid--level framework
		\end{itemize}
		
	\end{frame}
	
	\subsection{Yesod}
	
	\begin{frame}{Yesod}
		
		Advantages:
		\begin{itemize}
			\item Type safety in maximum form
			\item Easy to start (scaffolding, documentation in books and blogs)
			\item Packages are consistent; all from same developer
			\item Warp server, said to be the fastest
			\item Using Haskell's advanced language features extensively
		\end{itemize}
		\pause
		Disadvantages:	
		\begin{itemize}
			\item Alternative packages difficult to swap
			\item Comprehensive feature set can be expensive for small applications
			\item Abuse of language features
		\end{itemize}
		
	\end{frame}
		
	\section{Conclusion and future work}
	
	\begin{frame}{Conclusion and future work}
	All three frameworks contain different development philosophies: 

	\begin{enumerate}
		\item Happstack allows the programmer to choose among a very diverse set of tools while providing a good feature set out of the box.
		\pause \item Snap provides a minimal system and does not impose any limitations, but also requires the programmer to resort to external libraries for most of the required functionality.
		\pause \item Yesod provides most of the usual functionality and tries to do so consistently, but this also makes it more difficult to divert from the given programming style.
	\end{enumerate}
	\pause
	In general, though, most of the functionality implemented is framework--independent.
	\begin{itemize}
		\item Benchmark individual libraries per functionality to isolate and compare the cost of language features
	\end{itemize}
	\end{frame}
	
	\begin{frame}{Conclusion and future work (cont.)}
	\begin{itemize}
		\item Many libraries available for HTML generation but few libraries for CSS and JavaScript generation
			\begin{itemize}
			\item Happstack provides only a $ JMacro $ library wrapper, and Snap does not provide any CSS or JavaScript generation Snaplets.
			\end{itemize}
	
		\item Snap and Happstack could also support type--safe URL routing like Yesod's or $ web-routes $'
	
		\item Happstack could make good use of a project scaffolding and management tool (such as the ones found in both Yesod and Snap)

		\item More Snaplets available through Hackage
	
		\item Better control over static file sharing in the $ warp $ web server
	\end{itemize}
	\end{frame}
	
	\begin{frame}{Questions}
		\begin{center}
			\large{Thank you!}\\
			\huge{Questions?}
		\end{center}
	\end{frame}
\end{document}
