Developing web pages and web--based applications requires the programmer to deal with the stateless nature of the HTTP protocol.
Abstracting away from this lack of state by using web frameworks is the most common solution. 
In this paper, we presented three of the most promising frameworks for developing web pages in Haskell: Happstack, Snap and Yesod.
We conclude that all three contain different development philosophies: 

\begin{enumerate}
	\item Happstack allows the programmer to choose among a very diverse set of tools while providing a good feature set out of the box.
	\item Snap provides a minimal system and does not impose any limitations, but also requires the programmer to resort to external libraries for most of the required functionality.
    \item Lastly, Yesod provides most of the usual functionality and tries to do so consistently, but this also makes it more difficult to divert from the given programming style.
\end{enumerate}

In general, though, most of the functionality implemented in the frameworks is independent and can be ported to one another.

Because of the interchangeable nature of most of the features, we paid special attention to the specific implementation details of the main libraries.
Specifically, we noted that Haskell allows avoiding common web programming mistakes by implementing libraries that take advantage of the language's type--safety features.
Language extensions and techniques, such as monad transformers, type families, Template Haskell and Quasi-quotation make the programmer's life easier by reducing the probability of security holes or incomplete implementations and help with code reuse.

All the frameworks have quite active development communities, and provide some healthy competition and mutual aid by code and library sharing.
However, some framework--specific issues could use attention.
For instance, Happstack could make good use of a project scaffolding and management tool such as the ones found in both Yesod and Snap.
More Snaplets available on Hackage would make Snap far more useful and interesting for start--up projects that may require new or different functionality over time.
Snap's $ snap-auth $ package could benefit from using the $ authenticate $ library to allow for more authentication methods.
At the moment, Yesod relies on the web server's implementation for static file serving. 
More fine--grained control over this feature in its $ warp $ web server should be on its to--do list.

While there are many libraries available for HTML generation, libraries for CSS and JavaScript generation are rare.
Happstack provides only a $ JMacro $ library wrapper, and Snap does not provide any CSS or JavaScript generation Snaplets.
Snap and Happstack could also profit from Yesod's and $ web-routes $' approach to type--safe URL routing.

Because Snap profiles itself as a ``simple and fast'' web framework with a ``minimal HTTP API'', it seems to be an ideal candidate for a Haskell--based embedded HTTP server.

Finally, Happstack's $ acid-state $ library, an in--memory ACID data store, provides an interesting alternative to traditional back--ends.
Comparing this library's performance against more conventional SQL and NoSQL databases may prove to be very interesting.

It is also worth noting that several benchmarks show the frameworks to be quite fast. \cite{snap,warp}
However, some of the benchmarks seem to be contradictory and are often relatively shallow.
An extensive performance comparison can provide a more definite answer to the question of which web framework excels in what areas.