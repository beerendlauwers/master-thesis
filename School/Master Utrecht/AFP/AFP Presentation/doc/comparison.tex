Most frameworks present a subset of common features that aid the developer in solving the most common problems for developing web pages.
These problems range from dealing with path routing and serving files to handling session information.
Not every framework implements every feature, opting instead to use external, stand--alone libraries.
We first review the frameworks' implementation of common features in \sref{featureanalysis} before giving a general analysis of each web framework in \sref{frameworkanalysis}.

\subsection{Feature analysis}\label{featureanalysis}

\subsubsection{Path Routing}\label{pathrouting}

When serving web--pages from complex projects, the layout of the files on the project root do not necessarily map to the served URLs on a one--to--one basis.
Some URLs may even be dynamic, relying on some form of server state, query string or following complex pattern rules being processed after each request.
In general, techniques to facilitate returning the correct HTTP responses can be grouped under the label ``Path Routing''.

The path routing system in Happstack uses the $ ServerMonad $ and $ MonadPlus $ classes to allow for easy and clear creation of routes using guards.
$ ServerMonad $ is a specialized version the $ MonadReader $ class, and provides functions for modifying and reading an HTTP $ Request $.
Static elements are checked with the $ dir $ and $ dirs $ guards.

Parts of the URL can be extracted using the $ FromReqURI $ class: the $ path $ guard will capture a string, which can then be used in the generation of the response:
One can create a $ FromReqURI $ instance for an arbitrary Haskell data type, so that a value of the data type is returned in place of a string.

HTTP request methods (GET, POST, HEAD, etc.) can also be used to guard against, allowing for custom responses based on the request method.
Through use of the $ MatchMethod $ class, it is possible to match on several methods at once or provide a custom method matching function.

It is also possible to guard against the $ host $ header to allow multiple domains to be hosted on a single Happstack server.
Finally, guarding against the HTTP request itself is also possible using $ guardRq $.

In Snap, every Snaplet allows for the adding of routes relative to its own instantiation.
Those routes can point to sub--Snaplets or directories to be served directly.
On startup, the Snap core takes care of unifying the routes added in every instantiation into a single set of valid URLs to be served.
There is not, however, any sort of validation of the routes being added other than the verification of the correct instantiation of the sub--Snaplets.

Yesod first splits up a requested path into several pieces, splitting the URL at all forward slashes.
It will then attempt to match this list of split elements to the site map, and, if successful, call the appropriate $ Handler $ function.

For example, a simple site map could look like this:

\begin{lstlisting}
/person/#String	PersonR	GET POST
/year/#Int	YearR	GET POST DELETE
/page/faq	FaqR
\end{lstlisting}

The resulting data type would look something like this:

\begin{lstlisting}
data MyRoute = PersonR String
             | YearR Int
             | FaqR
\end{lstlisting}

Handling requests to these URLs would then require, at the very least, three handlers: $ handleFaqR $, $ handleYearR $ and $ handlePersonR $.
However, one may wish to call a different handler depending on the request method.
In the case of $ YearR $, the handlers would then be $ getYearR $, $ postYearR $ and $ deleteYearR $.

Like Happstack, Yesod allows for extracting a part of the URL and passing it to the $ Handler $.
The $ Handler $ monad is a monad transformer stack consisting of a $ Reader $ providing access to information about the request, a $ Writer $ for adding extra response headers, a $ State $ for session variables, and a $ MEither $ transformer for handling static content and sending error responses.

Similar to Happstack's $ WebMonad $, Yesod offers ``short--circuiting'' behaviour with the $ MEither $ monad.
The current computation is escaped and a response is returned to the user.

Besides built--in options, there is also the $ web-routes $ package, which allows for highly type--safe URL routing.
It is designed to be highly flexible: mapping URL types to strings and vice versa can be done via Quasi--quotation, Template Haskell, generic programming libraries, parser libraries or custom functions.
A data type is defined that represents all valid routes. Then, a $ PathInfo $ instance is derived using TH:

\begin{lstlisting}
data SiteMap = Index | Page Int deriving deriving (Eq, Ord, Read, Show, Data, Typeable)
$(derivePathInfo ''SiteMap)
\end{lstlisting}

The library's $ RouteT $ monad transformer is wrapped around the server monad. In the example code, we will use Happstack's server monad: $ ServerPartT IO $.
$ RouteT $ is a $ Reader $ monad that guards against incorrect URLs, because $ RouteT $ is parametrized by the URL data type $ SiteMap $.
The $ RouteT $ also provides the $ ShowURL $ class, which provides the $ showURL $ (URL data type to $ String $) and $ showURLParams $ (hostname, port, path prefix) functions.

After deriving the $ PathInfo $ instance, routing a path is trivial:

\begin{lstlisting}[mathescape=true]
route :: SiteMap $ \rightarrow $ RouteT SiteMap (ServerPartT IO) Response
route url =
    case url of
      Index     $ \rightarrow $ mainPage
      (Page id) $ \rightarrow $ selectPage id
\end{lstlisting}

\noindent where $ mainPage $ and $ selectPage $ are Happstack $ Response $ generators.

An interesting extension to $ web-routes $ is the $ web-routes-boomerang $ library.
This library allows one to define a single grammar that generates a parser and a printer for URLs.
Greater control can then be exerted over the appearance of URLs while retaining maintainability. 
             
\subsubsection{Serving Files}
Customized file and directory serving can make static text files look like HTML pages and provide custom layouts to directory indices.
Both Happstack and Snap provide granular functions for creating custom functions.

Happstack provides high flexibility for serving static files from disk. It uses its $ WebMonad $, $ ServerMonad $ and $ FilterMonad $ to do so.
The $ FilterMonad $ monad is used for manipulating $ Response $ values, such as setting response codes and content types. Creating your own response--generating function is also possible.
The $ WebMonad $ monad is also quite nice: it allows you to escape the current computation and immediately return a $ Response $.
This $ Response $ can have have a different content type from the original $ Response $ that was being built.

In order to serve a directory, one uses the guards described in \sref{pathrouting} together with the function $ serveDirectory $.
Rolling your own $ serveDirectory $ or $ serveFile $ function is possible by piecing together functions from the $ Happstack.Server.FileServe.BuildingBlocks $ module.

Yesod relies on the web server for serving static files from disk. It does provide some functions for looking up files and serving them, but creating your own functions will be harder.

Snap provides a utility module, $ Snap.Util.FileServe $, with many interesting functions, such as automatic and customized generation of directory index files and dynamic MIME--type handlers, which allow for things such as pretty--printing source code automatically.

\subsubsection{HTML Generation}
Some web frameworks originally developed their own HTML library: Yesod developed Hamlet; Snap developed Heist.
Currently, these libraries and several others can be used in every web framework with little to no extra code required.

Happstack ships with BlazeHtml and HSP (Haskell Server Pages), but also has boilerplate modules for Heist, Hamlet and HStringTemplate.
Snap uses Heist as its default HTML library, and Hamlet is still the default for Yesod.

An overview of the most popular libraries:

\begin{enumerate}
	\item BlazeHtml is a fast, combinator--based library that allows you to mix HTML tags with your Haskell code.
	\item Heist is an XML template engine that separates the XML templates from the Haskell code, improving maintainability.
    \item Hamlet is an HTML library that uses Quasi--Quotation to use HTML tags directly in the code. It is type--safe and helps to prevent XSS issues and 404s.
    \item HSP gives you the ability to embed XML tags inside your Haskell code, and can thus also generate XML code.
    \item HStringTemplate is based on Java's StringTemplate library, and allows for both embedding templates inside Haskell code as well as loading them from disk.
\end{enumerate}

While on--disk templates allow for fast changes, the speed of compiled code is lost.
The $ happstack--plugins $ package mitigates this by providing a system for automatic type--checking, recompilation and reloading of modules on a running server.

\subsubsection{JavaScript And CSS Generation}
The same techniques for HTML generation could also be used for JavaScript and CSS generation.

Yesod provides the $ Cassius $, $ Lucius $ and $ Julius $ quasi-quoters.
$ Cassius $ allows insertion of variables into CSS code, such as RGB values and URL's generated by Haskell. It uses whitespace rules instead of braces and semicolons.
$ Lucius $ uses the traditional CSS syntax, but retains $ Cassius $' insertion properties.
It also allow one to nest CSS blocks.
For example, here is a normal CSS file:

\begin{lstlisting}
.article p { text-indent: 2em; }
.article a { text-decoration: none; }
\end{lstlisting}

Compare this with the following:

\begin{lstlisting}
.article {
    p { text-indent: 2em; }
    a { text-decoration: none; }
}
\end{lstlisting}

Finally, $ Julius $ is a very simple JavaScript generator that, like $ Cassius $ and $ Lucius $, allows insertion of variables and URL's.
However, it also supports insertion of templates.

Happstack provides a boilerplate module for $ JMacro $, an indepedent library that is more ambitious than Yesod's $ Julius $.
Happstack does not have a CSS generator.

$ JMacro $ provides JavaScript syntax checking, insertion of Haskell values and use of Haskell techniques such as lambda expressions:

\begin{lstlisting}[mathescape=true]
// Functional style declarations:

// Cannot be partially applied
fun add x y $ \rightarrow $ x + y;
// Can/Must be partially applied
fun addTwo x $ \rightarrow \backslash $y -> x + y;  

// Traditional JavaScript
var addThenMult = mult( 2, add(2,3) );
// Haskell-styled function application
var multThenAdd = add 2 (mult 2 3);
\end{lstlisting}

It also provides automatic variable scoping: scoped variables will be renamed in the final JavaScript to prevent overlap with variables of the same name declared elsewhere.
Antiquotation allows one to use Haskell code directly within JavaScript code:

\begin{lstlisting}
let foo = 5 in renderJs [$jmacro|var x = `(foo)`;|]

//Result
var jmId_0;
jmId_0 = 5;
\end{lstlisting}

Snap does not have CSS or JavaScript generators, and does not appear to have a Snaplet for $ JMacro $.

\subsubsection{Parsing Request Data And Form Generation}
URLs with query strings, GET and POST requests and cookies contain information that has to be extracted.

Happstack provides the $ HasRqData $ monad to easily extract information from the query string, GET and POST submissions and cookies.
Its $ look $ function searches through the query string and form submissions, while $ lookCookie $ does the same for cookies.
In order to extract data from form submissions, the body must be decoded with $ decodeBody $, after which it is temporarily saved in a flat file, which is garbage collected afterwards.

The $ lookRead $ function uses the type system to parse a parameter into a Haskell value, and integrates seamlessly into the $ RqData $ error handling for when parsing fails.
A custom parsing function can be supplied with the function $ checkRq $, which also allows for setting custom error messages.
The function also allows you to enforce a specific subset of input (for example, only allowing Haskell $ Int $s between 1 and 10).
Optional parameters are provided with the $ optional $ function from $ Control.Applicative $.

Yesod uses the $ yesod-form $ package to generate HTML forms. The library integrates seamlessly into Yesod.
It is quite elaborate, featuring type--safe form generation, parsing submitted form data into Haskell data types, JavaScript validation code generation, and an applicative way of combining forms ($ AForms $) to create more complex ones.
Moreover, form generation can also be done monadically ($ MForms $), which gives the programmer complete control over the form's layout.
Finally, it is also possibly to have pure input forms ($ IForms $) which only parse data, but do not generate HTML (perhaps because the HTML code has been written by hand). 
Note that an $ IForm $ does not support client--side validation or error reporting.

The $ Digestive Functors $ library supports Happstack and Snap and is used to create composable HTML forms.
Yesod's form library used some concepts from the $ digestive-functors $ library.\cite{AnnouncYesod08}
The library is inspired by the $ Formlets $ library.

Simple form elements or ``formlets'' can be defined in terms of primitive input tags, after which more complex forms can be composed using the formlets.
An example would be a personal information form consisting of three fields:

\begin{lstlisting}
data Info = Person String String Int deriving (Show)
infoForm = Person <$> inputText Nothing <*> inputText Nothing <*> inputTextRead "Could not parse value" (Just 25)
\end{lstlisting}

Creating a form with three of these forms is then as simple as:

\begin{lstlisting}
data Group = Persons Info Info Info deriving (Show)
groupForm = Persons <$> infoForm <*> infoForm <*> infoForm
\end{lstlisting}

The $ FormState $ newtype is a $ State $ monad in which the ``formlets'' will be composed, and provides a unique identifier for each $ <input> $ tag, which is used for field--specific error reporting.

\subsubsection{Session Management}

Session management is a common way to uphold user state. 
Yesod's library $ clientsession $ is used by Snap to implement session management.
Happstack does not have an official package available, but there are a few external libraries available.

The majority of the libraries do not support saving values other than user name, password and expiry time in the session data.

Yesod's $ clientsession $ stores session data in an encrypted client--side cookie, which is sent along with each request.
They justify this overhead by noting that no server--side database communication is required, which means state servers are not required: adding new web servers poses no issues.
Aside from login information, $ clientsession $ provides a message system for a common use case: after a POST request, the user is redirected to a new page and simultaneously sent a success message.
The library also integrates with $ yesod-auth $'s ultimate destination design: when a user wishes to access a page and is requested to log in, he is redirected to his desired page after logging in.

Snap's $ snap-auth $ package uses $ clientsession $ as a base, but adds the ability to save extra data using the $ CookieSession $ Snap extension.
At the time of writing, this package was not yet available on Hackage.
The $ mysnapsession $ package contains a simplified version of $ snap-auth $, but also supports in--memory sessions. 
It also provides a continuation--based programming model suited to multiple--request stateful interactions using in--memory sessions.

For Happstack, there exists the $ happstack-auth $ and $ happstack-extra $ packages.
The first library supports simple sessions (login, logout, expiry) that are hashed with SHA512 and checked with a fingerprint (IP and user--agent).
The second provides many handy features, including session management. 
Only the session ID is saved in a cookie.
Other session data is saved in--memory.

\subsubsection{Authentication}

Related to session management, user authentication is also a staple of web development. 
Integration with OpenID, Facebook Connect and so on are commonly requested features.
The $ authenticate $ package, spun off from Yesod, forms the basis for both Yesod's and Happstack's authentication libraries.

The $ authenticate $ package supports BrowserId, Facebook Connect, Kerberos, OAuth, OpenID and rpxnow.
$ yesod-auth $ adds E--mail, Google Mail with OpenID and HashDB to that list.
$ happstack-authenticate $ adds support for multiple authentication methods per account, as well as multiple personalities per user account and themeable BlazeHtml--based templates.
Snap's $ snap-auth $ provides only basic user/password authentication functionality.

\subsubsection{Persistence}

Similarly to HTML generation, the developers of two of the frameworks created their own back--end for persistent data storage: $ happstack-state $ for Happstack and $ persistent $ library for Yesod.
$ happstack-state $ has been succeeded by $ acid-state $ and is now independent of Happstack, and is thus discussed at the bottom of this section.
It is still the recommended persistence package for Happstack.
Both $ acid-state $ and $ persistent $ are non--relational.

The $ persistent $ library allows one to store arbitrary Haskell values.
It does so by generating boilerplate code using a combination of Template Haskell and Quasi--quotation:

\begin{lstlisting}
mkPersist sqlSettings [persist|
Person
    name String
    age Int
|]
\end{lstlisting}

This code block will generate a representation that can be explained to a database backend (supported out of the box are MongoDB, PostgreSQL and SQLite), mapping the data type's parameters to $ PersistValue $s, which correlate to a single field in a database column.
These $ PersistValues $ are marshalled to a $ PersistField $ class instance, which can be correlated to a database column.
An entire database table is viewed as a $ PersistEntity $ class instance. Both classes make use of phantom types, whereas previously Template Haskell was used.
Finally, a primary key data type is generated for the $ Person $ data type: $ PersonId $.

Data migrations caused by database schema changes are mitigated using several helper functions available in $ persistent $.
Data type changes and additions of new fields and entities are done automatically. 
Field and entity renames and data removal are not automatic. 
The $ printMigration $ function will print all actions the migration system will do without actually performing them.

Insert, update and deletion queries can be written in Haskell themselves, and $ persistent $ provides several helper functions.
For example, this code block will create a query that returns all persons with an age of 22 in ascending order:

\begin{lstlisting}[mathescape=true]
persons $ \leftarrow $ selectList [Age ==. 22] [Asc Age]
\end{lstlisting}

Because every entity has its own primary key data type, extra type safety is achieved.
For example, a query that attempts to use a $ PersonId $ to select a $ Car $ value will result in a compile--time error, as a $ CarId $ would be required.

While defining a data type for use in a database, it is possible to specify uniqueness constraints, default values, nullability, foreign keys and custom table and column names.

Snap does not have an in--house persistence system. Snaplets are available for the $ HDBC $ and $ mongoDB $ packages, as well as the $ acid-state $ library mentioned below.

$ AcidState $ is a framework--independent package that supplies a MACID store: an in--Memory ACID data store that is capable of storing arbitrary Haskell values.
It does so by writing to file which functions were executed along with their parameters. Thus, recreating state simply means re--running the saved logs.
Its predecessor was $ Happstack-state $, which functions similarly, but which is only available for Happstack.
$ AcidState $ is platform--independent.
User error cannot lead to corruption of data: exceptions will abort a transaction, deleted or renamed functions will be caught during re--running, and the format of the state is checked for parsability.
$ AcidState $ makes heavy use of $ MVar $s to ensure ACIDity, and uses $ IORef $s for remote communication.
Use of the $ SafeCopy $ package allows for easy future extension of data structures saved in the MACID store, as well as serialization of the data using $ SafeCopy $'s $ deriveSafeCopy $.

Updating the data store is done via the $ Update $ monad, which can be seen as the $ State $ monad for use in an ACID--environment.
In the same way, the $ Query $ monad can be seen as the $ Reader $'s ACID--equivalent.
The TH--function $ makeAcidic $ will generate $ SafeCopy $ instances for the update and query functions.
It will also generate data types based on the names of the passed functions, which are used to ensure that updates and queries incompatible with the MACID store will result in a type checking error.

\subsubsection{Project management}

For starting new projects both Yesod and Snap provide useful programs.
Yesod provides a scaffolding tool that asks the user a few questions, after which it will generate a cabal package containing a skeleton Yesod project.
Snap has a few commands to initialize a bare--bones skeleton or a simple, fully working ``Hello World!'' example.
Happstack does not provide any tools to set up a server project.

\subsection{Framework analysis}\label{frameworkanalysis}

\subsubsection{Happstack}
Happstack profiles itself as a flexible and fast web framework, attempting to give the developer free reign over as many aspects as possible.
Small boilerplate modules allow Happstack to interface with external libraries without forcing a developer to use a specific library.
In--house features are designed to be as modular as possible to allow developers to roll their own functions.
Happstack provides an extensive ``Crash Course'' \cite{HappstackCrashcourse} that will help a developer learn Happstack by playing with the code.
The Crash Course goes to great lengths to explain Happstack's capabilities without introducing abstract or complex concepts.
Of course, more extensive documentation can be found in the framework's API.
Overall, Happstack is a good choice for the developer with a hacking mentality: checking to see what works well together and having the ability to swap out parts they don't like.
Alternatively, Happstack also functions reliably as an actual website, supporting the full spectrum of possibilities of a modern web framework.

\subsubsection{Snap}
Snap is a framework focused on the management of Snaplets and, as such, has a relatively simple feature set.
However, given its simplicity, it is also straightforward to wrap standalone libraries into a Snap project.
Snaplets allow for high levels of modularization as each Snaplet can be treated as an isolated application/component and can also be instantiated multiple times in the same application.
Its strength seems to be in the simplicity and solid base, and as of now, it is mostly a mid--level framework, providing the basics for a web server and extending functionality via its Snaplet concept, which can be thought of as similar to Java Servlets.
The web framework comes with a few Snaplets, and several others are available via their website or via Hackage.
Snap provides documentation for installing and using Snap as a simple web framework, as well as information about Snaplet creation and design. 
The framework's API is well documented.
Snap is of interest to developers who require a small and fast web server with a minimal footprint, while retaining the ability to extend it with features they require.
While no examples could be found of Snap being used as an embedded HTTP server, it appears to be a prime candidate for this role.

\subsubsection{Yesod}
Yesod achieves flexibility and high speed in a different manner from Happstack.
This framework uses strongly integrated libraries to ensure high levels of type safety.
While most of its libraries are usable in other web frameworks, the majority of Yesod's features were developed in--house.
It leverages this opportunity to provide a tightly integrated web framework that helps the developer to exploit Haskell's type system to ensure type--safe URL creation, HTML and form generation, database communication, and much more.
Similar to Happstack's Crash Course, Yesod provides a ``Web Framework Book'' \cite{YesodBook} that explains Yesod's inner workings in detail without elaborating upon the advanced Haskell features it uses.
The book's appendices goes into deeper detail into some of the designs Yesod employs, and the framework's API is also richly documented.
It is only recently that Yesod has received an actual web server component.
Before this, it used web servers based on the $ wai $ package.
Its $ warp $ package, also built upon the $ wai $ package, appears to be very fast, with preliminary results showing it to be more than twice as fast compared to Snap and Happstack. \cite{warp}
Yesod's capabilities will interest developers who are interested in highly performant, highly type--safe applications by leveraging Haskell's strong points.