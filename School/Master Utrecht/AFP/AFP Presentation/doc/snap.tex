Snap is a web framework launched to the public on May 2010 centered on the creation and use of modules called Snaplets.
The framework is composed of a core package with most of the common code and the API for writing Snaplets and Snaplet packages for wrapping access for individual functionality.

The top--level Snap web application initializes and parametrizes every Snaplet it will use in an initialization function.
On this initialization function, sub--Snaplets are created, routes are added to the project and any other setup activity is defined and a $ SnapletInit $ data type is returned encapsulating a record with all the immediate sub--Snaplets' handlers (of type $ Handler $).
This record is usually created with the $ makeLenses $ Template Haskell function.

It is through the created $ SnapletInit $ that the Snap application is initialized and served.
This function is also called automatically on reload.

\subsubsection{Snaplets}

Each application developed in Snap is usually composed by one or more different (sub) Snaplets but is also itself a Snaplet.
This allows for modularization as each Snaplet can be treated as an isolated application/component and can also be instantiated multiple times in the same application.
After being initialized, each Snaplet is used (possibly by other sub--Snaplets) through a $ Handler $.

The $ Handler $ has a $ MonadState $ instance that allows the user to use the regular $ get $ and $ set $ methods from the State monad.
It is worth mentioning that the states seen by all Snaplets, except for external interactions using IO (e.g. with databases or file system), are isolated per served request.

Snap has an API that focuses on allowing Snaplets to have access to environment and state data.
Most code that do not depend on either can be left outside any Snaplet (and, in fact, be used by any other Haskell web--framework).
