\documentclass[10pt]{beamer}
\usepackage[utf8]{inputenc}
\usepackage{amsmath}
\usepackage{amsfonts}
\usepackage{amssymb}
\usepackage{listings}
\usepackage{natbib}
\usetheme{Antibes}
\lstset{language=Haskell}
\lstset{breaklines=true}
\lstset{basicstyle=\scriptsize}

\newcommand{\newblock}{}
\newcommand{\sufficient}{\tiny{$ \square $}}

\author{
	Beerend Lauwers\\
	\and 
	Augusto Martin\\	
	\and
	Frank Wijmans\newline\\
	\hbox{\{B.Lauwers, A.PassalaquaMartins, F.S.Wijmans\}@students.uu.nl}
	\and
	\newline Utrecht University, The Netherlands}
\date{\today}

\title{Comparing \textbf{Generic-Deriving} and \textbf{Multiplate} to other Generic Programming Libraries in Haskell}

\begin{document}

\frame{\titlepage}

\begin{frame}{Outline}
\tableofcontents
\end{frame}

\section{Introduction}
\subsection*{Motivation}
\begin{frame}
Our main motivation for this project was:

How well do \textbf{Generic-Deriving} and \textbf{Multiplate} compare to other Generic Programming Libraries?
\end{frame}

\subsection*{Previously compared libraries}

\begin{frame}{Libraries}
The GP Bench discussed on \citet{Rodriguez:2008:art} already presents comparison of the following libraries:
\begin{small}
\begin{itemize}
\item \textbf{Lightweight Impl. of Generics and Dynamics} (LIGD)
\item \textbf{Polytypic programming in Haskell} (PolyLib)
\item \textbf{Scrap your boilerplate} (SYB)
\item \textbf{SYB, spine view variant} (Spine)
\item \textbf{SYB, extensible variant using typeclasses} (SYB3)
\item \textbf{Extensible and Modular Generics for the Masses} (EMGM)
\item \textbf{RepLib}
\item \textbf{Smash your boilerplate} (Smash)
\item \textbf{Uniplate}
\end{itemize}
\end{small}
Also, there is similar comparison information available about MultiRec in the PhD thesis of \citet{Rodriguez:2009:phd}.
\end{frame}

\begin{frame}{Project implementation}
To answer the question, we added Multiplate and Generic--Deriving to the GP Bench code used by \citet{Rodriguez:2008:art}.

We also updated the benchmark code and a subset of the libraries to work with \textbf{GHC 7.2.1}.

Namely, we updated the following libraries:
\begin{itemize}
\item \textbf{Scrap your boilerplate} (SYB)
\begin{itemize}
\item Included with GHC
\end{itemize}
\item \textbf{Extensible and Modular Generics for the Masses} (EMGM)
\begin{itemize}
\item Similar to generic-deriving
\end{itemize}
\item \textbf{Uniplate}
\begin{itemize}
\item Similar to Multiplate
\end{itemize}
\end{itemize}

\end{frame}

\subsection{Multiplate}
\begin{frame}{Multiplate}

\begin{itemize}
	\item Combination of Uniplate and Compos
	\item Native support for multi-type traversals for mutually recursive datatypes
	\item Only requires rank-3 polymorphism extension
\end{itemize}

\end{frame}

\subsubsection*{Example}
\begin{frame}[fragile]{Multiplate - Example}

\begin{lstlisting}
-- Simple mutually recursive language
 data Expr = Con Int
           | Add Expr Expr
           | EVar Var
           | Let Decl Expr
           deriving (Eq, Show)
 
 data Decl = Var := Expr
           | Seq Decl Decl
           deriving (Eq, Show)
 
 type Var = String
\end{lstlisting}

Define a $ Plate $ record type:
\begin{lstlisting}
-- Plate record type
 data Plate f = Plate
            { expr :: Expr -> f Expr
            , decl :: Decl -> f Decl
            }
\end{lstlisting}

\end{frame}


\begin{frame}[fragile]{Multiplate - Example}
Define an instance of $ Multiplate $ :
\begin{lstlisting}
 instance Multiplate Plate where
 multiplate child = Plate buildExpr buildDecl
   where
    buildExpr (Add e1 e2) = Add <$> expr child e1 <*> expr child e2
    buildExpr (Let d e) = Let <$> decl child d <*> expr child e
    buildExpr e = pure e
    buildDecl (v := e) = (:=) <$> pure v <*> expr child e
    buildDecl (Seq d1 d2) = Seq <$> decl child d1 <*> decl child d2
 mkPlate build = Plate (build expr) (build decl)
\end{lstlisting}

Boilerplate's done! Example traversal function:

\begin{lstlisting}
-- Collect all variable names
 varPlate :: Plate (Constant [Var])
 varPlate = purePlate { expr = exprVars }
  where
   exprVars (EVar v) = Constant [v]
   exprVars x = pure x
   
 collectVars :: Expr -> [Var]
 collectVars = foldFor expr $ preorderFold $ varPlate
\end{lstlisting}

\end{frame}

\subsection{Generic--Deriving}
\begin{frame}{Generic--Deriving}

\begin{itemize}
	\item Uses $ DeriveGeneric $ and $ DefaultSignatures $ introduced in GHC 7.2
	\item Very little boilerplate code required
	\item Drawback: only abstracts over single type parameter
\end{itemize}

\end{frame}

\subsubsection*{Example}
\begin{frame}[fragile]{Generic Deriving - Example}
Create a class that acts upon the structure representation:
\begin{lstlisting}
class GEq' f where
  geq' :: f a -> f a -> Bool

instance GEq' U1 where
  geq' _ _ = True

instance (GEq c) => GEq' (K1 i c) where
  geq' (K1 a) (K1 b) = geq a b

instance (GEq' a) => GEq' (M1 i c a) where
  geq' (M1 a) (M1 b) = geq' a b

instance (GEq' a, GEq' b) => GEq' (a :+: b) where
  geq' (L1 a) (L1 b) = geq' a b
  geq' (R1 a) (R1 b) = geq' a b
  geq' _      _      = False

instance (GEq' a, GEq' b) => GEq' (a :*: b) where
  geq' (a1 :*: b1) (a2 :*: b2) = geq' a1 a2 && geq' b1 b2
\end{lstlisting}

\end{frame}

\begin{frame}[fragile]{Generic Deriving - Example}
Create another class that acts as a mediator between the actual type and the generic function
\begin{lstlisting}
class GEq a where 
  geq :: a -> a -> Bool
  default geq :: (Generic a, GEq' (Rep a)) => a -> a -> Bool
  geq x y = geq' (from x) (from y)
\end{lstlisting}

Ad-hoc cases:
\begin{lstlisting}
instance GEq Char   
  where geq = (==)
instance GEq Int    
  where geq = (==)
instance GEq Float  
  where geq = (==)
\end{lstlisting}

Default cases:
\begin{lstlisting}
instance (GEq a) => GEq (Maybe a)
instance (GEq a) => GEq [a]
\end{lstlisting}

\end{frame}

\section{Evaluation criteria}
\begin{frame}{Evaluation criteria}

We evaluated the following criteria:
\begin{description}
\item[Types]
\begin{itemize}
\item Universe
\begin{itemize}
\item \textit{Which datatype classes are supported?}
\end{itemize}
\item Subuniverses
\begin{itemize}
\item \textit{Can generic functions be limited to certain datatypes?}
\end{itemize}
\end{itemize}
\end{description}

\begin{description}
\item[Expressiveness]
\begin{itemize}
\item First-class functions
\begin{itemize}
\item \textit{Can a generic function be passed to another generic function?}
\end{itemize}
\item Abstraction over type constructors 
\begin{itemize}
\item \textit{Is it possible to define generic map and generic fold functions?}
\end{itemize}
\item Separate compilation
\begin{itemize}
\item \textit{Do you have to recompile the library to extend the universe?}
\end{itemize}
\end{itemize}
\end{description}

\end{frame}
\begin{frame}{Evaluation criteria (Cont.)}
	
	We evaluated the following criteria:
\begin{description}
\item[Expressiveness]
\begin{itemize}
\item Ad-hoc definitions for constructors/datatypes	
\begin{itemize}
\item \textit{Is datatype-specific (non-generic) behaviour possible in a generic function?}
\end{itemize}
\item Extensibility
\begin{itemize}
\item \textit{Can you (non-generically, so ad-hoc required) extend a generic function in a different module?}
\end{itemize}
\item Multiple arguments
\begin{itemize}
\item \textit{Can a generic function have multiple generic arguments?}
\end{itemize}
\item Constructor names
\begin{itemize}
\item \textit{Can you add metadata to a type representation?}
\end{itemize}
\item Consumers, transformers and producers
\begin{itemize}
\item \textit{Can you define functions of types $ (a \rightarrow T) $, $ ( a \rightarrow a ) $, $ ( a \rightarrow b ) $ and , $ ( T \rightarrow a ) $?}
\end{itemize}
\end{itemize}
\end{description}
\end{frame}
\begin{frame}{Evaluation criteria (Cont.)}
	
	We evaluated the following criteria:
\begin{description}
\item[Usability]
\begin{itemize}
\item Portability
\begin{itemize}
\item \textit{How many compiler extensions does the library require and how common are they?}
\end{itemize}
\item Overhead of library use
\begin{itemize}
\item \textit{How much code do you have to write to get a type representation / generic function / generic function instance?}
\end{itemize}
\item Practical aspects
\begin{itemize}
\item \textit{How well is the library and its documentation maintained?}
\end{itemize}
\item Ease of learning/use.
\begin{itemize}
\item \textit{How difficult is it to understand the library's functionalities and mechanisms?}
\end{itemize}
\end{itemize}
\end{description}

\begin{description}
\item[Design Choices]
\begin{itemize}
\item Implementation mechanisms
\begin{itemize}
\item \textit{How are types represented at runtime?}
\end{itemize}
\item Views
\begin{itemize}
\item \textit{Which views does the library support (e.g. sum-of-products)?}
\end{itemize}
\end{itemize}
\end{description}

\end{frame}

\section{Comparison results}

\begin{frame}{Multiplate results}

(Only criteria with noteworthy results are discussed.)
\begin{description}
\item[Types]
\begin{itemize}
\item \textbf{Universe}: Defining nested datatypes was unclear
\end{itemize}
\end{description}

\begin{description}
\item[Expressiveness]
\begin{itemize}
\item \textbf{Separate compilation}: Supported through defining a $ Plate $ record type
\item \textbf{Ad-hoc definitions for constructors and datatypes}: Support via $ purePlate $ 
\item \textbf{Consumers, transformers and producers}: Does not support producer functions
\end{itemize}
\end{description}

\begin{description}
\item[Usability]
\begin{itemize}
\item \textbf{Portability}: Only rank-3 polymorphism required (could be brought down to rank-2 polymorphism)
\item \textbf{Overhead of library use}
\begin{itemize}
\item \textbf{Automatic generation of representations}: Unsupported, but possible through TH
\item \textbf{Work to instantiate a generic function}: Little compared to other libraries
\end{itemize}
\item \textbf{Practical aspects}: No longer actively maintained
\end{itemize}
\end{description}

\end{frame}

\begin{frame}{Generic-Deriving results}

(Only criteria with noteworthy results are discussed.)
\begin{description}
\item[Types]
\begin{itemize}
\item \textbf{Universe}: Abstracts over a single type parameter
\end{itemize}
\end{description}

\begin{description}
\item[Expressiveness]
\begin{itemize}
\item \textbf{First-class functions}: Currently unsupported, research ongoing
\item \textbf{Abstraction over type constructors}: $ crushRight $ instance for composition limited functionality, nested trees didn't terminate
\end{itemize}
\end{description}

\begin{description}
\item[Usability]
\begin{itemize}
\item \textbf{Portability}: Only multi-parameter type classes
\item \textbf{Overhead of library use}
\begin{itemize}
\item \textbf{Automatic generation of representations}: $ Generic $ is derived by compiler, $ Generic1 $ possible through TH
\item \textbf{Work to instantiate a generic function}: Default class methods minimized overhead
\end{itemize}
\end{itemize}
\end{description}

\end{frame}

\begin{frame}{Comparison table}


\begin{table}[ht]
	\tiny
		
\begin{tabular}{l | c c c c c c}
& \begin{tiny}
SYB
\end{tiny} & \begin{tiny}
EMGM
\end{tiny} & \begin{tiny}
Uniplate
\end{tiny} & \begin{tiny}
Multiplate
\end{tiny} & \begin{tiny}
Multirec
\end{tiny} & \begin{tiny}
Generic Deriving
\end{tiny} \\
\hline
Universe Size & & & & & & \\
\begin{tiny}
Regular datatypes
\end{tiny} & $ \bullet $ & $ \bullet $ & $ \bullet $ & $ \bullet $ & $\bullet$ & $ \bullet $ \\
\begin{tiny}
Higher-kinded datatypes
\end{tiny} & $ \bullet $ & $ \bullet $ & $ \bullet $ & $ \circ $ & $ \circ $ & $ \bullet $ \\
\begin{tiny}
Nested datatypes
\end{tiny} & $ \bullet $ & $ \bullet $ & $ \bullet $ & $ \circ $ & $ \circ $ & $ \bullet $ \\
\begin{tiny}
Nested \& higher-kinded
\end{tiny} & $ \circ $ & \sufficient & $ \circ $ & $ \circ $ & $ \circ $ & $ \circ $ \\
\begin{tiny}
Mutually recursive
\end{tiny} & $ \bullet $ & $ \bullet $ & $ \bullet $ & $ \bullet $ & $ \bullet $ & $ \bullet $ \\
Subuniverses & $ \circ $ & $ \bullet $ & $ \circ $ & $ \circ $ & $ \circ $ & $ \bullet $ \\
\hline
First-class generic functions & $ \bullet $ & \sufficient & $ \circ $ & $ \circ $ & $ \circ $ & $ \circ $ \\
Abstraction over type constructors & $ \circ $ & $ \bullet $ & $ \circ $ & $ \circ $ & $ \circ $ & $ \bullet $ \\
Separate compilation & $ \bullet $ & $ \bullet $ & $ \bullet $ & $ \bullet $ & $ \bullet $ & $ \bullet $ \\
Ad-hoc definitions for datatypes & $ \bullet $ & $ \bullet $ & $ \bullet $ & $ \bullet $ & \sufficient & $ \bullet $ \\
Ad-hoc definitions for constructors & $ \bullet $ & $ \bullet $ & $ \bullet $ & $ \bullet $ & $ \bullet $ & $ \bullet $ \\
Extensibility & $ \circ $ & $ \bullet $ & $ \circ $ & $ \circ $ & $ \circ $ & $ \bullet $ \\
Multiple arguments & \sufficient & $ \bullet $ & $ \circ $ & $ \circ $ & $ \bullet $ & $ \bullet $ \\
Constructor names & $ \bullet $ & $ \bullet $ & $ \circ $ & $ \circ $ & $ \bullet $ & $ \bullet $ \\
Consumers & $ \bullet $ & $ \bullet $ & $ \bullet $ & $ \bullet $ & $ \bullet $ & $ \bullet $ \\
Transformers & $ \bullet $ & $ \bullet $ & $ \bullet $ & \sufficient & $ \bullet $ & $ \bullet $ \\
Producers & \sufficient & $ \bullet $ & $ \circ $ & $ \circ $ & $ \bullet $ & $ \bullet $ \\
\hline
Portability & $ \circ $ & $ \bullet $ & $ \bullet $ & $ \bullet $ & $ \circ $ & $ \bullet $ \\
Overhead of library use & & & & & &  \\
\begin{tiny}
Automatic generation of representations
\end{tiny} & $ \bullet $ & $ \circ $ & $ \bullet $ & $ \circ $ & \sufficient & \sufficient \\
\begin{tiny}
Number of structure representations
\end{tiny} & $ 2 $ & $ 4 $ & $ 1 $ & $ 1 $ & $ 1 $ & $ 2 $ \\
\begin{tiny}
Work to instantiate a generic function
\end{tiny} & $ \bullet $ & \sufficient & $ \bullet $ & $ \bullet $ & $ \bullet $ & $ \bullet $ \\
\begin{tiny}
Work to define a generic function
\end{tiny} & $ \bullet $ & $ \bullet $ & $ \bullet $ & $ \bullet $ & $ \bullet $ & $ \bullet $ \\
Practical aspects & $ \bullet $ & $ \circ $ & $ \bullet $ & $ \circ $ & $ \bullet $ & $ \bullet $ \\
Ease of learning and use & $ \circ $ & \sufficient & $ \bullet $ & $ \bullet $ & $ \circ $ & $ \bullet $ \\
\hline
& \begin{tiny}
SYB
\end{tiny} & \begin{tiny}
EMGM
\end{tiny} & \begin{tiny}
Uniplate
\end{tiny} & \begin{tiny}
Multiplate
\end{tiny} & \begin{tiny}
Multirec
\end{tiny} & \begin{tiny}
Generic Deriving
\end{tiny} \\
\end{tabular}
	\caption{Evaluation of generic programming libraries}
\end{table}

\end{frame}

\section{Conclusions}
\begin{frame}{Conclusions}
To conclude what we have done:
\begin{itemize}
	\item Updated a existing benchmark to compile under \textbf{GHC 7.2.1}
	\item Added code for Generic Deriving and Multiplate to the benchmark
	\item Compared the results for EMGM, SYB, Uniplate, Multiplate, Generic Deriving and MultiRec
\end{itemize}

Our conclusions:
\begin{itemize}
	\item Multiplate could serve as a starting point for specialized library
	\begin{itemize}
	\item Minimal boilerplate code
	\item Relatively easy to use
	\item Possibly much faster compared to MultiRec
	\end{itemize}
	\item Generic--Deriving looks promising
	\begin{itemize}
	\item Automatic deriving limits boilerplate code and complexity
	\item Supports many datatypes
	\item Author mentioned further enhancements (\citet{generic-deriving:2010:lib})
	\end{itemize}
\end{itemize}
\end{frame}

\subsection{Future work}
\begin{frame}{Future work}
	\begin{description}
		\item[Performance testing] Define criteria for performance testing and test and compare all libraries accordingly
		\item[Cabal package] Create a cabal package from the benchmark code to make it easier for others to (re)use it
	\end{description}
\end{frame}

\begin{frame}{References}
\bibliographystyle{plainnat}
\bibliography{doc}
\end{frame}

\end{document}
