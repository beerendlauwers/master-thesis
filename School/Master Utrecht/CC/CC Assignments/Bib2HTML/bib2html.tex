\documentclass[10pt,a4paper]{article}
\usepackage[latin1]{inputenc}
\usepackage{amsmath}
\usepackage{amsfonts}
\usepackage{amssymb}
\usepackage{listings}

\author{Beerend Lauwers and Frank Wijmans}
\title{Building a BibTeX-to-HTML compiler}
\begin{document}
\section{Data types}	
	Haskell data type:

   	 \begin{lstlisting}
    data Bibtex   = BibTex [Entry]
    data Entry    = Entry Type Title [Attr]
    data Type     = Book | Inproceedings | Article | ..
    type Title    = String
    data Attr     = Attr AttrType String
    data AttrType = Author | Title | Year | ..	
    \end{lstlisting}
	
	In an ATerm it would look like:
	
	\texttt{List [} 
	
	\texttt{App ``Entry'' [ App ``Book'', }
		
	\texttt{String ``This is the books title.'', }
		
	\texttt{[ App ``Attr'' [ App ``Year'' [] , String ``2002''  ] ] ] }
	
	\texttt{]}
	
	
	\section{Proposal}
	
	Three elements, as given in the project description.
    
Firstly, parse a string to the datatype above. We might consider parsing directly into the $ ATerm $ datatype to speed up this process.
	This seems to be a straight-forward computation using the datatype as a red line through the parser.
    If we haven't done so directly, build an $ ATerm $ from that datatype.
\newline\newline
Secondly, the input (an $ ATerm $) is converted back into a $ BibTex $ value and checked for rules given in the LaTeX documentation. 
	Why convert it back? $ \rightarrow $ Added type safety, especially handy for type of entry and attributes types.
	Finally, convert the checked $ BibTex $ database into an HTML-format $ ATerm $. Also, we sort the entries by type, and sort entry attributes as well.
\newline\newline
Lastly, pretty-print the $ ATerm $ so that we output an  HTML representation.
	HTML is tree-like structured, so every node gets three lines, one for each begin and end tag, and a line for the inner value.
	Every leaf would be printed on one line.
	In a simple example:
\newline\newline
	\texttt{<p>}\newline
	\texttt{<b>Bolded text in a paragraph</b>}\newline
	\texttt{</p>}\newline

	\subsection{Excluded functionality}
	Certain BibTeX functionality, like some lenient notation and non-standard characters, might be troublesome to parse in Haskell. We choose to first get minimal functionality, and afterwards add some corner cases to the existing code.
	
	
	
\end{document}